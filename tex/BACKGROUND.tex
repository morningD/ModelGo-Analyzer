\section{Background and Related Work}
\label{sec:related}

\subsection{License Compliance Analysis}
The previous license compliance analysis studies primarily focus on OSS-licensed software~\cite{schoettle2019open, cui2023empirical}, and several successful tools, such as FOSSology~\cite{jaeger2017fossology} and Black Duck Software Composition Analysis~\cite{blackduck2024}, have been developed.
The main goal of these tools is to identify all open-source dependencies in software projects to evaluate associated license compliance risks, obligations, and attribution requirements.
Typically, the component dependencies and license information in a software project can be obtained through scanning and matching~\cite{ombredanne2020free}. 
This process involves gathering information from notices, headers, licenses, and other project files or attempting to match the code to determine its provenance.
Unfortunately, these strategies cannot be naturally extended to ML projects for the following reasons.

First, the components of ML projects, particularly models, have more intricate dependencies than code.
For example, knowledge can be transferred between models without explicitly copying weights~\cite{you2021workshop}.
Second, the OSS Bill of Materials standard, such as Software Package Data Exchange (SPDX)~\cite{spdx2024}, does not fully support common model licenses like OpenRAIL-M~\cite{contractor2022behavioral}, Llama 2~\cite{meta2024llama2}.
Third, non-standard licensing (model publishing under OSS or free-content licenses) is prevalent in ML projects~\cite{duan2024modelgo}, adding complexity to license analysis.
Furthermore, the crowd-sourced nature of ML components may also lead to over-permissive licenses~\cite{rajbahadur2021can}, distorting the analysis results.

For these reasons, license compliance issues in the ML field remain nearly unexplored. Rajbahadur \textit{et al.}~\cite{rajbahadur2021can} investigated license provenance issues in ML datasets and found instances of non-compliance between their licenses and the licenses of their data sources.
Building on these findings, Duan \textit{et al.}~\cite{duan2024modelgo} proposed a tool that analyzes conflicts directly based on the licenses of data sources and provided guidelines to minimize such conflicts.
However, their tool is limited to generating analysis reports and lacks the ability to visualize and exchange ML workflows, making it difficult to extend or integrate external resources from the web (e.g., linking to another workflow via URI).
Therefore, in this work, we propose an ontology describing ML workflows using RDF graphs (a visualized workflow is provided in Appendix~\ref{apdx:grapher}), making it both extensible and linkable, and then analyze license compliance through an automated reasoning engine.

\subsection{Model Licensing}
Today, we have many OSS licenses to accommodate diverse publishing scenarios~\cite{rosen2005open}. However, do they still function as intended in model publishing scenarios? The answer is no. While these licenses aim to govern the use and distribution of software, they lack definitions of ML concepts, which compromises their effectiveness (ref. Section~\ref{sec:license}).
Some model licenses (or agreements) are also emerging, such as Llama 2 and Gemma~\cite{gemma2024}. 
However, most of these licenses are specifically designed to govern certain models or their derivatives and are not as open as they claim to be~\cite{liesenfeld2024rethinking}. 
Meanwhile, Contractor \textit{et al.}~\cite{contractor2022behavioral} proposed OpenRAIL-M, a model license derived from Apache-2.0. 
While this license offers good clarity, it enforces use behavior restrictions that render it non-compliant with open-source licenses like GPL-3.0~\cite{greenbaum2016the}.
In addition, although OpenRAIL-M has many variations, its license terms are quite homogenized and lack the flexibility needed to accommodate different model publishing scenarios, such as non-commercial use, open sourcing, and restrictions on sharing outputs. 
Therefore, a significant number of developers have opted to publish their models using CC licenses, such as CC-BY-NC-4.0, as an alternative to prohibit commercial use of their models.
However, these licenses also face the issue of losing effectiveness in the context of model publishing.
Such unstandardized licensing practices can lead to increased compliance issues and pose potential legal hazards in ML projects~\cite{duan2024modelgo}. 
To promote standardized model publication, we propose a new set of model licenses that provide a wider range of licensing options.