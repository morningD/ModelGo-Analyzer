\section{Experiments}
\label{sec:case}
This section seeks to answer a key question: \emph{Should I continue using traditional OSS and free-content licenses to publish my model, and what are the associated risks?}
To explore this question, we evaluate commonly adopted licenses in the context of model publishing by MG Analyzer to assess whether they are still effective as intended.
The example workflow involves combining two models and publishing them as a service, as shown in Figure~\ref{fig:case1}.
Here, we consider two scenarios: 1) publishing the combined model as a service; 2) publishing the data generated from the combined model.
Models A and B are non-binding works that correspond to the settings in Table~\ref{tab:result}, with their respective analysis results also presented in the table.
Model C is an intermediate work created by combining A and B, and Data D is the generated output from C. 
Work E involves republishing C as a service with the intent to sell, while Work F involves republishing D as a literary form, also with the intent to sell.
To be more convincing, we use real-world models and their respective licenses for demonstration.

\begin{figure}[t]
    \centering
    \includegraphics{fig/case1.pdf}
    \caption{Example Workflow: Combine Models, Then Publish.}
    \Description{}
    \label{fig:case1}
\end{figure}

\begin{table}[]
    \caption{Settings and Analysis Results.}
    \footnotesize
    \begin{tabular}{|ll|}
    \hline \rowcolor[gray]{.8}
    \multicolumn{1}{|l|}{Work: License} & Work: Report code. \\ \hline
    \multicolumn{1}{|p{3.2cm}|}{(i) C: AGPL-3.0. D: Unlicense. \newline (ii) D: Unlicense } & \multicolumn{1}{p{4.5cm}|}{ (i) E: N1N2N3$\times$2; \textbf{W5}$\times$2; W1$\times$3. F: W1$\times$3. \newline (ii) E: W1$\times$2. F: W1.} \\ \hline
    \multicolumn{2}{|p{8cm}|}{\textbf{OSS License Setting:} \newline
    (i) A$\leftarrow$PhoBERT~\cite{nguyen2020phobert}: AGPL-3.0. B$\leftarrow$CKIP-Transformers~\cite{lin2022hantrans}: GPL-3.0. \newline
    (ii) A$\leftarrow$PhoBERT~\cite{nguyen2020phobert}: AGPL-3.0. B$\leftarrow$\emph{None}.
    }\\ \hline \hline

    \multicolumn{1}{|p{3.2cm}|}{C: Unlicense. D: Unlicense.} & E: W1$\times$2, \textbf{E2}$\times$2. F: W1$\times$2 . \\ \hline
    \multicolumn{2}{|p{8cm}|}{\textbf{Free-content License Setting:}\newline
        A$\leftarrow$ MPT-Chat~\cite{MosaicML2023Introducing}: CC-BY-NC-SA-4.0.  B$\leftarrow$Command R+~\cite{cohereforai2024c4ai}: CC-BY-NC-4.0.
    } \\ \hline \hline

    \multicolumn{1}{|p{3.2cm}|}{(iii) D: Unlicense. \newline (iv) D: Unlicense.} & \multicolumn{1}{p{4.5cm}|}{ (iii) E: N1; N2: \textbf{W5} . F: \emph{None}. \newline (iv) E: N1; N2; W2$\times$2; \textbf{E2}; \textbf{E5}. F: W2, \textbf{E5}.} \\ \hline
    \multicolumn{2}{|p{8cm}|}{\textbf{Model License Setting:}\newline
        (iii) A$\leftarrow$ MG-BY-OS. B$\leftarrow$\emph{None}. (iv) A$\leftarrow$ MG-BY-NC. B$\leftarrow$\emph{None}.
    } \\ \hline
    \end{tabular}
    \label{tab:result}
\end{table}

First, we evaluate two OSS licenses: AGPL-3.0 and GPL-3.0, which are considered enforceable open-source licenses with copyleft clauses.
In the setting (i), Work E triggers two \emph{Disclose Source Code} warnings (code W5, refer to Table~\ref{tab:report} for code definitions), and AGPL-3.0 successfully proliferates to Work C.
However, with a small adjustment, we can circumvent these clauses by republishing the generated content rather than providing the model as a service. 
As demonstrated by Work D, there is no W5 warning, and the content is licensed under the Unlicense.
Furthermore, the condition in AGPL-3.0 that triggers the disclose code clauses related to remote access is \emph{you modify the Program}, which means you can directly republish copies as a service to circumvent this clause.
As reflected in setting (ii), there are no more W5 warnings, only two warnings related to the non-standard licensing remain.



