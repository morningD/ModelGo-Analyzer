%%
%% This is file `sample-sigconf.tex',
%% generated with the docstrip utility.
%%
%% The original source files were:
%%
%% samples.dtx  (with options: `all,proceedings,bibtex,sigconf')
%% 
%% IMPORTANT NOTICE:
%% 
%% For the copyright see the source file.
%% 
%% Any modified versions of this file must be renamed
%% with new filenames distinct from sample-sigconf.tex.
%% 
%% For distribution of the original source see the terms
%% for copying and modification in the file samples.dtx.
%% 
%% This generated file may be distributed as long as the
%% original source files, as listed above, are part of the
%% same distribution. (The sources need not necessarily be
%% in the same archive or directory.)
%%
%%
%% Commands for TeXCount
%TC:macro \cite [option:text,text]
%TC:macro \citep [option:text,text]
%TC:macro \citet [option:text,text]
%TC:envir table 0 1
%TC:envir table* 0 1
%TC:envir tabular [ignore] word
%TC:envir displaymath 0 word
%TC:envir math 0 word
%TC:envir comment 0 0
%%
%%
%% The first command in your LaTeX source must be the \documentclass
%% command.
%%
%% For submission and review of your manuscript please change the
%% command to \documentclass[manuscript, screen, review]{acmart}.
%%
%% When submitting camera ready or to TAPS, please change the command
%% to \documentclass[sigconf]{acmart} or whichever template is required
%% for your publication.
%%
%%
\documentclass[sigconf, anonymous, review]{acmart}
%\documentclass[sigconf]{acmart}
%%
%% \BibTeX command to typeset BibTeX logo in the docs
\AtBeginDocument{%
  \providecommand\BibTeX{{%
    Bib\TeX}}}

%% Rights management information.  This information is sent to you
%% when you complete the rights form.  These commands have SAMPLE
%% values in them; it is your responsibility as an author to replace
%% the commands and values with those provided to you when you
%% complete the rights form.
\setcopyright{acmlicensed}
\copyrightyear{2018}
\acmYear{2018}
\acmDOI{XXXXXXX.XXXXXXX}

%% These commands are for a PROCEEDINGS abstract or paper.
\acmConference[Conference acronym 'XX]{Make sure to enter the correct
  conference title from your rights confirmation emai}{June 03--05,
  2018}{Woodstock, NY}
%%
%%  Uncomment \acmBooktitle if the title of the proceedings is different
%%  from ``Proceedings of ...''!
%%
%%\acmBooktitle{Woodstock '18: ACM Symposium on Neural Gaze Detection,
%%  June 03--05, 2018, Woodstock, NY}
\acmISBN{978-1-4503-XXXX-X/18/06}


%%
%% Submission ID.
%% Use this when submitting an article to a sponsored event. You'll
%% receive a unique submission ID from the organizers
%% of the event, and this ID should be used as the parameter to this command.
%%\acmSubmissionID{123-A56-BU3}

%%
%% For managing citations, it is recommended to use bibliography
%% files in BibTeX format.
%%
%% You can then either use BibTeX with the ACM-Reference-Format style,
%% or BibLaTeX with the acmnumeric or acmauthoryear sytles, that include
%% support for advanced citation of software artefact from the
%% biblatex-software package, also separately available on CTAN.
%%
%% Look at the sample-*-biblatex.tex files for templates showcasing
%% the biblatex styles.
%%

%%
%% The majority of ACM publications use numbered citations and
%% references.  The command \citestyle{authoryear} switches to the
%% "author year" style.
%%
%% If you are preparing content for an event
%% sponsored by ACM SIGGRAPH, you must use the "author year" style of
%% citations and references.
%% Uncommenting
%% the next command will enable that style.
%%\citestyle{acmauthoryear}
\usepackage{comment}
\usepackage{multirow}
\usepackage{amsmath,pifont}
\usepackage{booktabs}
\usepackage{rotating}
\usepackage{array}
\usepackage[para]{footmisc}
\usepackage{tablefootnote}
\usepackage{xcolor}
\usepackage{colortbl}
\usepackage{longtable}
\usepackage{circledsteps}
\usepackage{romannum}
\usepackage{longtable}
\usepackage{tcolorbox}
\usepackage{enumitem} % no indent [leftmargin=*]
\usepackage{balance}

\definecolor{Permissive}{HTML}{18AFCC}
\definecolor{Public}{HTML}{5F9099}
\definecolor{Copyleft}{HTML}{CC1865}

\hyphenation{OpenAI Model-Go} 

%%
%% end of the preamble, start of the body of the document source.
\begin{document}

%%
%% The "title" command has an optional parameter,
%% allowing the author to define a "short title" to be used in page headers.
\title{"They've Stolen My GPL-Licensed Model!": Toward Standardized and Transparent Model Licensing}

%%
%% The "author" command and its associated commands are used to define
%% the authors and their affiliations.
%% Of note is the shared affiliation of the first two authors, and the
%% "authornote" and "authornotemark" commands
%% used to denote shared contribution to the research.
\author{Ben Trovato}
\authornote{Both authors contributed equally to this research.}
\email{trovato@corporation.com}
\orcid{1234-5678-9012}
\author{G.K.M. Tobin}
\authornotemark[1]
\email{webmaster@marysville-ohio.com}
\affiliation{%
  \institution{Institute for Clarity in Documentation}
  \city{Dublin}
  \state{Ohio}
  \country{USA}
}

\author{Lars Th{\o}rv{\"a}ld}
\affiliation{%
  \institution{The Th{\o}rv{\"a}ld Group}
  \city{Hekla}
  \country{Iceland}}
\email{larst@affiliation.org}

\author{Valerie B\'eranger}
\affiliation{%
  \institution{Inria Paris-Rocquencourt}
  \city{Rocquencourt}
  \country{France}
}

\author{Aparna Patel}
\affiliation{%
 \institution{Rajiv Gandhi University}
 \city{Doimukh}
 \state{Arunachal Pradesh}
 \country{India}}

\author{Huifen Chan}
\affiliation{%
  \institution{Tsinghua University}
  \city{Haidian Qu}
  \state{Beijing Shi}
  \country{China}}

\author{Charles Palmer}
\affiliation{%
  \institution{Palmer Research Laboratories}
  \city{San Antonio}
  \state{Texas}
  \country{USA}}
\email{cpalmer@prl.com}

\author{John Smith}
\affiliation{%
  \institution{The Th{\o}rv{\"a}ld Group}
  \city{Hekla}
  \country{Iceland}}
\email{jsmith@affiliation.org}

\author{Julius P. Kumquat}
\affiliation{%
  \institution{The Kumquat Consortium}
  \city{New York}
  \country{USA}}
\email{jpkumquat@consortium.net}

%%
%% By default, the full list of authors will be used in the page
%% headers. Often, this list is too long, and will overlap
%% other information printed in the page headers. This command allows
%% the author to define a more concise list
%% of authors' names for this purpose.
\renewcommand{\shortauthors}{Trovato et al.}

%%
%% The abstract is a short summary of the work to be presented in the
%% article.
\begin{abstract}
  As model parameter sizes reach the billion-level range and their training consumes zettaFLOPs of computation, components reuse and collaborative development are become increasingly prevalent in the machine learning (ML) community.
  These components, including models, software, and datasets, may originate from various sources and be published under different licenses, which govern the use and distribution of licensed materials and their derivatives.
  However, commonly chosen licenses, such as GPL and Apache, are software-specific and may not be clearly defined and bounded in the context of model publishing.
  Meanwhile, the reused components may also have free-content licenses and model licenses, which pose a potential risk of license conflicts and rights infringement within the model production workflow.
  In this paper, we propose addressing the above challenges along two lines:
    1) For license analysis, we have developed a new vocabulary for ML workflow management and encoded license rules to enable ontological reasoning for analyzing rights granting and license conflicts.
    2) For standardized model publishing, we have drafted a set of model licenses that provide flexible options to meet the diverse needs of model publishing.
    Our analysis tool is built on Turtle language and the Notation 3 reasoning engine, serving as a first step toward Linked Open Model Production Data.
    We have also encoded our proposed model license set into rules and demonstrated the effects of GPL and other commonly used licenses in model publishing, along with the flexibility advantages of our proposed model licenses, through experiments.

\end{abstract}

% revise: However, the common choosed licenses, such as GPL and Apache, are software-specific and may not enforce in the machine learning context like publisher expect, expose a risk of 
%. This can expose licensed models to risks that are beyond the publisher's desired control

\begin{comment}
  Productionizing machine learning projects is inherently complex, involving a multitude of interconnected components that are assembled like LEGO blocks and evolve throughout development lifecycle.
  These components encompass software, databases, and models, each subject to various licenses governing their reuse and redistribution.
  However, existing license analysis approaches for Open Source Software (OSS) are not well-suited for this context.
  For instance, some projects are licensed without explicitly granting sublicensing rights, or the granted rights can be revoked, potentially exposing their derivatives to legal risks.
  Indeed, the analysis of licenses in machine learning projects grows significantly more intricate as it involves interactions among diverse types of licenses and licensed materials.
  To the best of our knowledge, no prior research has delved into the exploration of license conflicts within this domain.
  In this paper, we introduce ModelGo, a practical tool for auditing potential legal risks in machine learning projects to enhance compliance and fairness.
  With ModelGo, we present license assessment reports based on five use cases with diverse model-reusing scenarios, rendered by real-world machine learning components.
  Finally, we summarize the reasons behind license conflicts and provide guidelines for minimizing them.
  Our code is publicly available at \url{https://github.com/Xtra-Computing/ModelGo}.
\end{comment}

%%
%% The code below is generated by the tool at http://dl.acm.org/ccs.cfm.
%% Please copy and paste the code instead of the example below.
%%
\begin{CCSXML}
<ccs2012>
 <concept>
  <concept_id>00000000.0000000.0000000</concept_id>
  <concept_desc>Do Not Use This Code, Generate the Correct Terms for Your Paper</concept_desc>
  <concept_significance>500</concept_significance>
 </concept>
 <concept>
  <concept_id>00000000.00000000.00000000</concept_id>
  <concept_desc>Do Not Use This Code, Generate the Correct Terms for Your Paper</concept_desc>
  <concept_significance>300</concept_significance>
 </concept>
 <concept>
  <concept_id>00000000.00000000.00000000</concept_id>
  <concept_desc>Do Not Use This Code, Generate the Correct Terms for Your Paper</concept_desc>
  <concept_significance>100</concept_significance>
 </concept>
 <concept>
  <concept_id>00000000.00000000.00000000</concept_id>
  <concept_desc>Do Not Use This Code, Generate the Correct Terms for Your Paper</concept_desc>
  <concept_significance>100</concept_significance>
 </concept>
</ccs2012>
\end{CCSXML}

\ccsdesc[500]{Do Not Use This Code~Generate the Correct Terms for Your Paper}
\ccsdesc[300]{Do Not Use This Code~Generate the Correct Terms for Your Paper}
\ccsdesc{Do Not Use This Code~Generate the Correct Terms for Your Paper}
\ccsdesc[100]{Do Not Use This Code~Generate the Correct Terms for Your Paper}

%%
%% Keywords. The author(s) should pick words that accurately describe
%% the work being presented. Separate the keywords with commas.
\keywords{License Analysis, AI Licensing, Automated Reasoning}

\received{20 February 2007}
\received[revised]{12 March 2009}
\received[accepted]{5 June 2009}

%%
%% This command processes the author and affiliation and title
%% information and builds the first part of the formatted document.
\maketitle

tbd~\cite{huang2022layoutlmv3}
%\section{Introduction}
In recent years, the compelling generalization capabilities provided by billion-parameter models~\cite{wei2023inverse}, along with the high computational and data costs associated with their training~\cite{maslej2024aiindex}, have motivated ML project developers to collaborate incrementally rather than train models from scratch.
For example, a common approach is to download a Pre-Trained Model (PTM)~\cite{jiang2023empirical} and fine-tune it for downstream task~\cite{hu2022lora}.
However, these paradigms may face potential legal risks if the use and redistribution practices violate the governing licenses of the reused components, akin to the GPL violation issues in the field of Open Source Software (OSS)~\cite{mathur2012empirical}.
Another risk arises from the choice of license used to republish the work. Some developers adhere to traditional software publishing practices and select OSS licenses for their models~\cite{devlin2019bert, ni2022expanding}, which often lack clear definitions and conditions regarding ML activities and do not effectively prevent undesirable use.
For example, a licensee can close-source your published models, even if they are licensed under GPL, without violating any terms.

There are three possible ways to address above challenges.
First, developers could avoid using any third-party materials. However, this is extremely difficult for individual developers, as training PTMs is expensive and requires vast amounts of data. For instance, the training dataset for GPT-2~\cite{radford2019language} was collected from 45 million web pages, governed by various licenses and terms of use.
Second, a new licensing framework for ML projects could be developed, which might include drafting specific licenses for models and datasets~\cite{benjamin2019towards, contractor2022behavioral}, along with a compatibility table to guide their reuse policies. 
However, this approach also has limitations, as it does little to address existing conflicts in ML projects that rely on components released under traditional licenses. Furthermore, it is impractical to expect all publishers to relicense their previous works.
Third, we can scan the reused components in ML projects and analyze existing license compliance risks to eliminate them. 
This is a common solution applied to OSS projects~\cite{jaeger2017fossology} but it cannot be directly extended to ML projects.
The reason is that ML components can involve complex coupling mechanisms and different licensing frameworks that are interwoven within a project.

Take MixLoRA~\cite{li2024mixlora} as an example: it is licensed under Apache-2.0 (an OSS license) and is fine-tuned on Llama2~\cite{touvron2023llama} (governed by Llama2 Community License~\cite{meta2023llama2}, a model license) using the Cleaned Alpaca Dataset~\cite{taori2023alpaca}, which licensed under CC BY-NC-4.0 (a free content license from Creative Commons~\cite{creative2023list}).
Previous OSS license analysis tools~\cite{mathur2012empirical, ombredanne2020free} that only consider package reference dependencies and focus on software licenses will fall short in such ML scenarios, which involve implicit nested dependencies and various licensing frameworks.
Therefore, to provide license analysis for ML projects, the key is to develop an interpretative solution that can cover all licensing frameworks and disambiguate their mapping rules related to ML activities.
Moreover, the lack of consensus in standard model publishing practices and the inflexibility of available model licenses have led many developers to publish their models under OSS licenses or even free-content licenses~\cite{groeneveld2024olmo, cohereforai2024c4ai}, further increasing the complexity of designing license analysis methods.

In this paper, we propose a two-pronged approach to address these challenges. 
First, to resolve existing license conflicts in ML projects, we introduce \textit{MG Analyzer}, a tool that constructs ML workflows as Resource Description Framework (RDF) graphs and assesses potential license compliance issues, improper license selection, rights grants, restrictions, and obligations within the projects. 
Second, to promote standardized model publishing in the future, we propose a new set of model-specific licenses, \textit{MG Licenses}, offering Creative Commons-style licensing options for developers to choose from.
To present potential risks of using traditional OSS, model, and free-content licenses in model publishing scenarios, we present XXXXXXXX use case.
We also demonstrate the flexibility of \textit{MG Licenses} in encompassing nearly all licensing conditions provided by other model licenses (by our analyzer?).
The main contributions of our paper are:

\begin{itemize}
    \item Our code and license text are available at (link temporarily hidden due to double-blind policy).
\end{itemize}

The rest of the paper is organized as follows. 


\begin{table*}[h]
    \caption{Licenses.}
    \tiny
    \begin{tabular}{|lcccccccccccccccc|}
        \hline
        \multicolumn{1}{|l|}{\multirow{2}{*}{License Name}} & \multicolumn{4}{c|}{Clarity of Definitions} & \multicolumn{3}{c|}{Freedom of Verbatim Copy} & \multicolumn{5}{c|}{Freedom of Derivative} & \multicolumn{2}{c|}{Freedom of Use} & \multicolumn{1}{c|}{\multirow{2}{*}{Clarity}} & \multirow{2}{*}{Freedom} \\ \cline{2-15}

        \multicolumn{1}{|l|}{} & \multicolumn{1}{l|}{Prefixes} & \multicolumn{1}{l|}{Rights} & \multicolumn{1}{l|}{Rules} & \multicolumn{1}{l|}{Remote} & \multicolumn{1}{l|}{Share} & \multicolumn{1}{l|}{Close} & \multicolumn{1}{l|}{Non-excl} & \multicolumn{1}{l|}{Share} & \multicolumn{1}{l|}{Close} & \multicolumn{1}{l|}{Non-excl} & \multicolumn{1}{l|}{Sublicense} & \multicolumn{1}{l|}{Attribute} & \multicolumn{1}{l|}{Comm} & \multicolumn{1}{l|}{Behav} & \multicolumn{1}{l|}{} & \\ 
        \hline
        
        \multicolumn{1}{|r|}{Apache-2.0} & \cmark & \cmark & \pmark & \multicolumn{1}{c|}{\xmark} & \pmark & \cmark & \multicolumn{1}{c|}{\cmark} & \pmark & \pmark &\cmark & \cmark & \multicolumn{1}{c|}{\xmark} & \cmark & \multicolumn{1}{c|}{\cmark} & \multicolumn{1}{c|}{2.5} & 7.5 \\

        \multicolumn{1}{|r|}{MIT} & \pmark & \pmark & \pmark & \multicolumn{1}{c|}{\xmark} & \pmark & \cmark & \multicolumn{1}{c|}{\cmark} & \cmark & \cmark & \cmark & \cmark & \multicolumn{1}{c|}{\xmark} & \cmark & \multicolumn{1}{c|}{\cmark} & \multicolumn{1}{c|}{1.5} & 8.5 \\

        \multicolumn{1}{|r|}{Unlicensed} & \pmark & \pmark & \pmark & \multicolumn{1}{c|}{\xmark} & \cmark & \cmark & \multicolumn{1}{c|}{\cmark} & \cmark & \cmark & \cmark & \cmark & \multicolumn{1}{c|}{\cmark} & \cmark & \multicolumn{1}{c|}{\cmark} & \multicolumn{1}{c|}{1.5} & 10 \\

        \multicolumn{1}{|r|}{AFL-3.0} & \pmark & \cmark & \pmark & \multicolumn{1}{c|}{\cmark} & \pmark & \cmark & \multicolumn{1}{c|}{\cmark} & \pmark & \pmark & \cmark & \cmark & \multicolumn{1}{c|}{\xmark} & \cmark & \multicolumn{1}{c|}{\cmark} & \multicolumn{1}{c|}{3.0} & 7.5 \\

        \multicolumn{1}{|r|}{Artistic-2.0} & \pmark & \pmark & \pmark & \multicolumn{1}{c|}{\cmark} & \pmark & \xmark & \multicolumn{1}{c|}{\cmark} & \pmark & \xmark & \cmark & \cmark & \multicolumn{1}{c|}{\xmark} & \cmark & \multicolumn{1}{c|}{\cmark} & \multicolumn{1}{c|}{2.5} & 6.0 \\

        \multicolumn{1}{|r|}{GPL-3.0} & \cmark & \cmark & \pmark & \multicolumn{1}{c|}{\xmark} & \pmark & \xmark & \multicolumn{1}{c|}{\pmark} & \pmark & \xmark & \pmark & \pmark & \multicolumn{1}{c|}{\xmark} & \cmark & \multicolumn{1}{c|}{\cmark} & \multicolumn{1}{c|}{2.5} & 4.5 \\

        \multicolumn{1}{|r|}{GPL-2.0} & \xmark & \cmark & \pmark & \multicolumn{1}{c|}{\xmark} & \pmark & \xmark & \multicolumn{1}{c|}{\cmark} & \pmark & \xmark & \cmark & \pmark & \multicolumn{1}{c|}{\xmark} & \cmark & \multicolumn{1}{c|}{\cmark} & \multicolumn{1}{c|}{1.5} & 5.5 \\

        \multicolumn{1}{|r|}{AGPL-3.0} & \cmark & \cmark & \pmark & \multicolumn{1}{c|}{\cmark} & \pmark & \xmark & \multicolumn{1}{c|}{\pmark} & \pmark & \xmark & \pmark & \pmark & \multicolumn{1}{c|}{\xmark} & \cmark & \multicolumn{1}{c|}{\cmark} & \multicolumn{1}{c|}{3.5} & 4.5 \\

        \multicolumn{1}{|r|}{LGPL-3.0} & \cmark & \cmark & \pmark & \multicolumn{1}{c|}{\xmark} & \pmark & \xmark & \multicolumn{1}{c|}{\cmark} & \pmark & \cmark & \cmark & \pmark & \multicolumn{1}{c|}{\cmark} & \cmark & \multicolumn{1}{c|}{\cmark} & \multicolumn{1}{c|}{2.5} & 7.5 \\

        \multicolumn{1}{|r|}{LGPL-2.1} & \xmark & \cmark & \pmark & \multicolumn{1}{c|}{\xmark} & \pmark & \xmark & \multicolumn{1}{c|}{\cmark} & \pmark & \xmark & \cmark & \pmark & \multicolumn{1}{c|}{\xmark} & \cmark & \multicolumn{1}{c|}{\cmark} & \multicolumn{1}{c|}{1.5} & 5.5 \\
        
        \multicolumn{1}{|r|}{BSD-3-Clause} & \xmark & \pmark & \pmark & \multicolumn{1}{c|}{\xmark} & \pmark & \cmark & \multicolumn{1}{c|}{\cmark} & \pmark & \cmark & \cmark & \cmark & \multicolumn{1}{c|}{\xmark} & \pmark & \multicolumn{1}{c|}{\cmark} & \multicolumn{1}{c|}{1.0} & 7.5 \\

        \multicolumn{1}{|r|}{BSD-3-Clause-Clear} & \xmark & \pmark & \pmark & \multicolumn{1}{c|}{\xmark} & \pmark & \cmark & \multicolumn{1}{c|}{\cmark} & \pmark & \cmark & \cmark & \cmark & \multicolumn{1}{c|}{\xmark} & \pmark & \multicolumn{1}{c|}{\cmark} & \multicolumn{1}{c|}{1.0} & 7.5 \\

        \multicolumn{1}{|r|}{BSD-2-Clause} & \xmark & \pmark & \pmark & \multicolumn{1}{c|}{\xmark} & \pmark & \cmark & \multicolumn{1}{c|}{\cmark} & \pmark & \cmark & \cmark & \cmark & \multicolumn{1}{c|}{\xmark} & \pmark & \multicolumn{1}{c|}{\cmark} & \multicolumn{1}{c|}{1.0} & 7.5 \\

        \multicolumn{1}{|r|}{WTFPL-2.0} & \xmark & \xmark & \xmark & \multicolumn{1}{c|}{\xmark} & \cmark & \cmark & \multicolumn{1}{c|}{\cmark} & \cmark & \cmark & \cmark & \cmark & \multicolumn{1}{c|}{\cmark} & \cmark & \multicolumn{1}{c|}{\cmark} & \multicolumn{1}{c|}{0} & 10 \\

        \multicolumn{1}{|r|}{OSL-3.0} & \pmark & \cmark & \pmark & \multicolumn{1}{c|}{\cmark} & \pmark & \cmark & \multicolumn{1}{c|}{\xmark} & \pmark & \xmark & \xmark & \cmark & \multicolumn{1}{c|}{\xmark} & \cmark & \multicolumn{1}{c|}{\cmark} & \multicolumn{1}{c|}{3.0} & 5 \\

        \multicolumn{1}{|r|}{ECL-2.0} & \pmark & \cmark & \pmark & \multicolumn{1}{c|}{\xmark} & \pmark & \cmark & \multicolumn{1}{c|}{\cmark} & \pmark & \pmark & \cmark & \cmark & \multicolumn{1}{c|}{\xmark} & \cmark & \multicolumn{1}{c|}{\cmark} & \multicolumn{1}{c|}{2.0} & 7.5 \\ 
        \hline

        \multicolumn{1}{|r|}{CC0-1.0} & \cmark & \cmark & \cmark & \multicolumn{1}{c|}{\xmark} & \cmark & \cmark & \multicolumn{1}{c|}{\cmark} & \cmark & \cmark & \cmark & \cmark * & \multicolumn{1}{c|}{\cmark} & \cmark & \multicolumn{1}{c|}{\cmark} & \multicolumn{1}{c|}{3.0} & 10 \\

        \multicolumn{1}{|r|}{CC-BY-4.0} & \cmark & \cmark & \pmark & \multicolumn{1}{c|}{\xmark} & \pmark & \cmark & \multicolumn{1}{c|}{\pmark} & \pmark & \pmark & \pmark & \pmark & \multicolumn{1}{c|}{\xmark} & \cmark & \multicolumn{1}{c|}{\cmark} & \multicolumn{1}{c|}{2.5} & 6 \\

        \multicolumn{1}{|r|}{CC-BY-SA-4.0} & \cmark & \cmark & \pmark & \multicolumn{1}{c|}{\xmark} & \pmark & \cmark & \multicolumn{1}{c|}{\pmark} & \pmark & \pmark & \pmark & \pmark & \multicolumn{1}{c|}{\xmark} & \cmark & \multicolumn{1}{c|}{\cmark} & \multicolumn{1}{c|}{2.5} & 6 \\

        \multicolumn{1}{|r|}{CC-BY-NC-4.0} & \cmark & \cmark & \pmark & \multicolumn{1}{c|}{\xmark} & \pmark & \cmark & \multicolumn{1}{c|}{\pmark} & \pmark & \pmark & \pmark & \pmark & \multicolumn{1}{c|}{\xmark} & \xmark & \multicolumn{1}{c|}{\cmark} & \multicolumn{1}{c|}{2.5} & 5 \\

        \multicolumn{1}{|r|}{CC-BY-ND-4.0} & \cmark & \cmark & \pmark & \multicolumn{1}{c|}{\xmark} & \pmark & \cmark & \multicolumn{1}{c|}{\pmark} & \xmark & \namark ** & \namark & \namark & \multicolumn{1}{c|}{\namark} & \cmark & \multicolumn{1}{c|}{\cmark} & \multicolumn{1}{c|}{2.5} & 4 \\

        \multicolumn{1}{|r|}{CC-BY-NC-ND-4.0} & \cmark & \cmark & \pmark & \multicolumn{1}{c|}{\xmark} & \pmark & \cmark & \multicolumn{1}{c|}{\pmark} & \xmark & \namark ** & \namark & \namark & \multicolumn{1}{c|}{\namark} & \xmark & \multicolumn{1}{c|}{\cmark} & \multicolumn{1}{c|}{2.5} & 3 \\

        \multicolumn{1}{|r|}{CC-BY-NC-SA-4.0} & \cmark & \cmark & \pmark & \multicolumn{1}{c|}{\xmark} & \pmark & \cmark & \multicolumn{1}{c|}{\pmark} & \pmark & \pmark & \pmark & \pmark & \multicolumn{1}{c|}{\xmark} & \xmark & \multicolumn{1}{c|}{\cmark} & \multicolumn{1}{c|}{2.5} & 5 \\

        \multicolumn{1}{|r|}{PDDL-1.0} & \cmark & \pmark & \cmark & \multicolumn{1}{c|}{\xmark} & \cmark & \cmark & \multicolumn{1}{c|}{\cmark} & \cmark & \cmark & \cmark & \cmark & \multicolumn{1}{c|}{\cmark} & \cmark & \multicolumn{1}{c|}{\cmark} & \multicolumn{1}{c|}{2.5} & 10 \\

        \multicolumn{1}{|r|}{C-UDA} & \cmark & \xmark & \pmark & \multicolumn{1}{c|}{\xmark} & \cmark & \cmark & \multicolumn{1}{c|}{\cmark} & \pmark & \cmark & \cmark & \xmark & \multicolumn{1}{c|}{\xmark} & \xmark & \multicolumn{1}{c|}{\xmark} & \multicolumn{1}{c|}{1.5} & 5.5 \\

        \multicolumn{1}{|r|}{LGPLLR} & \xmark & \cmark & \pmark & \multicolumn{1}{c|}{\xmark} & \pmark & \cmark & \multicolumn{1}{c|}{\xmark} & \pmark & \pmark & \xmark & \pmark & \multicolumn{1}{c|}{\xmark} & \pmark & \multicolumn{1}{c|}{\cmark} & \multicolumn{1}{c|}{1.5} & 4.5 \\

        \multicolumn{1}{|r|}{GFDL} & \cmark & \pmark & \pmark & \multicolumn{1}{c|}{\xmark} & \pmark & \xmark & \multicolumn{1}{c|}{\xmark} & \pmark & \xmark & \xmark & \pmark & \multicolumn{1}{c|}{\xmark} & \cmark & \multicolumn{1}{c|}{\cmark} & \multicolumn{1}{c|}{2.0} & 3.5 \\

        \multicolumn{1}{|r|}{ODC-By-1.0} & \pmark & \pmark & \cmark & \multicolumn{1}{c|}{\cmark} & \pmark & \cmark & \multicolumn{1}{c|}{\cmark} & \pmark & \cmark & \cmark & \pmark & \multicolumn{1}{c|}{\xmark} & \cmark & \multicolumn{1}{c|}{\cmark} & \multicolumn{1}{c|}{3.0} & 7.5 \\
        \hline

        \multicolumn{1}{|r|}{{\ddag}~OpenRAIL-M} & \cmark & \cmark & \cmark & \multicolumn{1}{c|}{\cmark} & \pmark & \cmark & \multicolumn{1}{c|}{\cmark} & \pmark & \pmark & \cmark & \cmark & \multicolumn{1}{c|}{\xmark} & \cmark & \multicolumn{1}{c|}{\xmark} & \multicolumn{1}{c|}{4.0} & 6.5 \\

        \multicolumn{1}{|r|}{{\dag}~Llama2} & \pmark & \xmark & \cmark & \multicolumn{1}{c|}{\xmark} & \pmark & \cmark & \multicolumn{1}{c|}{\cmark} & \pmark & \cmark & \cmark & \xmark & \multicolumn{1}{c|}{\xmark} & \pmark & \multicolumn{1}{c|}{\xmark} & \multicolumn{1}{c|}{1.5} & 5.5 \\

        \multicolumn{1}{|r|}{{\dag}~Llama3} & \pmark & \xmark & \cmark & \multicolumn{1}{c|}{\cmark} & \pmark & \cmark & \multicolumn{1}{c|}{\cmark} & \pmark & \cmark & \cmark & \xmark & \multicolumn{1}{c|}{\xmark} & \pmark & \multicolumn{1}{c|}{\xmark} & \multicolumn{1}{c|}{2.5} & 5.5 \\

        \multicolumn{1}{|r|}{{\dag}~Llama3.1} & \pmark & \xmark & \cmark & \multicolumn{1}{c|}{\cmark} & \pmark & \cmark & \multicolumn{1}{c|}{\cmark} & \pmark & \cmark & \cmark & \xmark & \multicolumn{1}{c|}{\xmark} & \pmark & \multicolumn{1}{c|}{\xmark} & \multicolumn{1}{c|}{2.5} & 5.5 \\

        \multicolumn{1}{|r|}{{\dag}~Gemma} & \xmark & \xmark & \cmark & \multicolumn{1}{c|}{\cmark} & \pmark & \cmark & \multicolumn{1}{c|}{\cmark} & \pmark & \pmark & \cmark & \xmark & \multicolumn{1}{c|}{\xmark} & \pmark & \multicolumn{1}{c|}{\xmark} & \multicolumn{1}{c|}{2.0} & 4.5 \\

        \multicolumn{1}{|r|}{MG0} & \cmark & \cmark & \cmark & \multicolumn{1}{c|}{\cmark} & \pmark & \cmark & \multicolumn{1}{c|}{\cmark} & \pmark & \cmark & \cmark & \cmark & \multicolumn{1}{c|}{\pmark} & \cmark & \multicolumn{1}{c|}{\cmark} & \multicolumn{1}{c|}{4.0} & 8.5 \\

        \multicolumn{1}{|r|}{MG-BY} & \cmark & \cmark & \cmark & \multicolumn{1}{c|}{\cmark} & \pmark & \cmark & \multicolumn{1}{c|}{\cmark} & \pmark & \pmark & \cmark & \cmark & \multicolumn{1}{c|}{\xmark} & \cmark & \multicolumn{1}{c|}{\cmark} & \multicolumn{1}{c|}{4.0} & 7.5 \\

        \multicolumn{1}{|r|}{MG-BY-RAI} & \cmark & \cmark & \cmark & \multicolumn{1}{c|}{\cmark} & \pmark & \cmark & \multicolumn{1}{c|}{\cmark} & \pmark & \pmark & \cmark & \cmark & \multicolumn{1}{c|}{\xmark} & \cmark & \multicolumn{1}{c|}{\xmark} & \multicolumn{1}{c|}{4.0} & 6.5 \\

        \multicolumn{1}{|r|}{MG-BY-OS} & \cmark & \cmark & \cmark & \multicolumn{1}{c|}{\cmark} & \pmark & \xmark & \multicolumn{1}{c|}{\cmark} & \pmark & \xmark & \cmark & \cmark & \multicolumn{1}{c|}{\xmark} & \cmark & \multicolumn{1}{c|}{\cmark} & \multicolumn{1}{c|}{4.0} & 6.0 \\

        \multicolumn{1}{|r|}{MG-BY-NC} & \cmark & \cmark & \cmark & \multicolumn{1}{c|}{\cmark} & \pmark & \cmark & \multicolumn{1}{c|}{\cmark} & \pmark & \cmark & \cmark & \pmark & \multicolumn{1}{c|}{\xmark} & \xmark & \multicolumn{1}{c|}{\cmark} & \multicolumn{1}{c|}{4.0} & 6.5 \\

        \multicolumn{1}{|r|}{MG-BY-NC-RAI} & \cmark & \cmark & \cmark & \multicolumn{1}{c|}{\cmark} & \pmark & \cmark & \multicolumn{1}{c|}{\cmark} & \pmark & \cmark & \cmark & \pmark & \multicolumn{1}{c|}{\xmark} & \xmark & \multicolumn{1}{c|}{\xmark} & \multicolumn{1}{c|}{4.0} & 5.5 \\

        \multicolumn{1}{|r|}{MG-BY-NC-OS} & \cmark & \cmark & \cmark & \multicolumn{1}{c|}{\cmark} & \pmark & \xmark & \multicolumn{1}{c|}{\cmark} & \pmark & \xmark & \cmark & \pmark & \multicolumn{1}{c|}{\xmark} & \xmark & \multicolumn{1}{c|}{\cmark} & \multicolumn{1}{c|}{4.0} & 4.5 \\

        \multicolumn{1}{|r|}{MG-BY-ND} & \cmark & \cmark & \cmark & \multicolumn{1}{c|}{\cmark} & \pmark & \cmark & \multicolumn{1}{c|}{\cmark} & \xmark & \namark & \namark & \namark & \multicolumn{1}{c|}{\namark} & \cmark & \multicolumn{1}{c|}{\cmark} & \multicolumn{1}{c|}{4.0} & 4.5 \\

        \multicolumn{1}{|r|}{MG-BY-NC-ND} & \cmark & \cmark & \cmark & \multicolumn{1}{c|}{\cmark} & \pmark & \cmark & \multicolumn{1}{c|}{\cmark} & \xmark & \namark & \namark & \namark & \multicolumn{1}{c|}{\namark} & \xmark & \multicolumn{1}{c|}{\cmark} & \multicolumn{1}{c|}{4.0} & 3.5 \\

        \multicolumn{1}{|r|}{{\dag}~OPT-175B} & \cmark & \pmark & \cmark & \multicolumn{1}{c|}{\cmark} & \pmark & \cmark & \multicolumn{1}{c|}{\cmark} & \pmark & \cmark & \cmark & \xmark & \multicolumn{1}{c|}{\xmark} & \xmark & \multicolumn{1}{c|}{\xmark} & \multicolumn{1}{c|}{3.5} & 5.0 \\

        \multicolumn{1}{|r|}{{\dag}~{\textbullet}~AI2-ImpACT-LR} & \pmark & \xmark & \cmark & \multicolumn{1}{c|}{\cmark} & \pmark & \cmark & \multicolumn{1}{c|}{\cmark} & \pmark & \cmark & \cmark & \xmark & \multicolumn{1}{c|}{\xmark} & \pmark & \multicolumn{1}{c|}{\xmark} & \multicolumn{1}{c|}{2.5} & 5.5 \\

        \multicolumn{1}{|r|}{{\dag}~{\textbullet}~AI2-ImpACT-MR} & \pmark & \xmark & \cmark & \multicolumn{1}{c|}{\cmark} & \xmark & \namark & \multicolumn{1}{c|}{\namark} & \pmark & \cmark & \cmark & \xmark & \multicolumn{1}{c|}{\xmark} & \xmark & \multicolumn{1}{c|}{\xmark} & \multicolumn{1}{c|}{2.5} & 2.0 \\

        \multicolumn{1}{|r|}{{\dag}~{\textbullet}~AI2-ImpACT-HR} & \pmark & \xmark & \cmark & \multicolumn{1}{c|}{\cmark} & \xmark & \namark & \multicolumn{1}{c|}{\namark} & \xmark & \namark & \namark & \namark & \multicolumn{1}{c|}{\namark} & \xmark & \multicolumn{1}{c|}{\xmark} & \multicolumn{1}{c|}{2.5} & 0 \\

        \hline\hline
        \multicolumn{17}{|p{16cm}|}{
            \textbf{Header Definitions:} \newline
            \textbf{Prefixes}: \ding{51} The license explicitly includes sufficient prefixes that clearly describe scope and conditions of granting rights (e.g., revocable, sublicensable); $\bm{\approx}$ Some important prefixes are indeterminate; \ding{55} No prefixes are declared. \newline
            \textbf{Rights}: \ding{51} The license explicitly declares whether a patent license or a copyright license is granted; $\bm{\approx}$ Only the granting of a patent license or copyright license is stated; \ding{55} No explicit grant of either is provided. \newline
            \textbf{Rules}: \ding{51} The license terms cover all ML activities listed in (TBD); $\bm{\approx}$ Some ML activities fall outside the definition of this license; \ding{55} Almost no rules are set forth. \newline
            \textbf{Remote}: \ding{51} The license considers remote access situations (e.g., via API, Web, SaaS); \ding{55} No definitions or rules regarding remote access behaviors are set forth.\newline
            \textbf{Share}: \ding{51} The license permits the sharing of verbatim copies/derivatives created by you without any restrictions; $\bm{\approx}$ Some restrictions apply to sharing; \ding{55} Sharing verbatim copies/derivatives is prohibited. \newline
            \textbf{Close}: \ding{51} The license does not require you to disclose the source files of verbatim copies/derivatives created by you; $\bm{\approx}$ Modification statements are required; \ding{55} You must disclose the source files of your created copies/derivatives. \newline
            \textbf{Non-excl}usive: \ding{51} The license does not restrict you from adding new terms when republishing; $\bm{\approx}$ Certain types of terms are prohibited in republishing; \ding{55} All republishing must adhere to the original terms and conditions. \newline
            \textbf{Sublicense}: \ding{51} The license explicitly grants sublicensing rights; $\bm{\approx}$ The license prohibits sublicensing but offers automatic licensing instead; \ding{55} Sublicensing is either prohibited or not explicitly permitted. \newline
            \textbf{Attribute}: \ding{51} The license does not require retaining the original attribution and licenses in redistributed derivatives; $\bm{\approx}$ Attribution or license must be retained; \ding{55} Redistributed derivatives must retain the  attributions and licenses. \newline
            \textbf{Comm}ercial: \ding{51} The license explicitly grants commercial rights; $\bm{\approx}$ Commercial rights are not explicitly granted but not reserved either, or compromised commercial rights are granted; \ding{55} Commercial rights are reserved. \newline
            \textbf{Behav}ioral: \ding{51} The license does not restrict user behaviors; $\bm{\approx}$ Includes runtime controls (e.g., forced updates); \ding{55} Certain behaviors involving the licensed materials or derivatives are prohibited (e.g., harming, medical advice). \newline          
            \textbf{Clarity}/\textbf{Freedom} Score: \ding{51} $+1.0$, $\bm{\approx}$ $+0.5$, \ding{55} $+0$, n/a: $+0$ . Maximum Clarity Score: 4.0, Maximum Freedom Score: 10. \newline
            \textbf{Explanations:} \newline
            *~Although CC0-1.0 explicitly states that sublicensing is not allowed, sublicensing becomes unnecessary due to the Waiver of Rights. \newline
            **~Since CC-BY-ND-4.0 and CC-BY-NC-ND-4.0 prohibit the sharing of derivatives, judgments regarding redistributed derivatives are marked as "n/a" in the table. \newline
            \dag~These licenses (or terms of use, or agreements) are specifically drafted for certain products and are not intended for general model publishing purposes. \newline
            \ddag~As there are no fundamental differences between CreativeML Open RAIL-M, OpenRAIL++-M, BigCode Open RAIL-M, BigScience RAIL, and BigScience Open RAIL-M, these licenses are grouped under OpenRAIL-M. \newline
            \textbullet~We have used an archive of the AI2 ImpaACT license; the version is 2.0, with an effective date of January 8, 2024.
        } \\ \hline
    \end{tabular}
\end{table*}

\begin{comment}
  \begin{table}[h]
    \caption{Specifications of AI components used in case studies, which include \textcolor{Copyleft}{Copyleft License}, \textcolor{Permissive}{Permissive License}, \textcolor{Public}{Public Domain License} and Non-Public License.}
    \vspace{-1mm}
    \footnotesize
    \label{tab:works}
    \begin{tabular}{|p{1.6cm}|p{3cm}|p{0.6cm}|p{1.7cm}|}
        \hline
        \rowcolor[gray]{.8}
        \textbf{Work Name} & \textbf{License Name} & \textbf{Type} & \textbf{Modality/Usage}  \\ \hline
        Wikipedia & \textcolor{Copyleft}{CC-BY-SA-4.0} & \multirow{15}{*}{Data} & \multirow{6}{*}{Text}   \\ \cline{1-2}
        StackExchange & \textcolor{Copyleft}{CC-BY-SA-4.0}  &  &    \\ \cline{1-2}
        FreeLaw & CC-BY-ND-4.0 &  &   \\ \cline{1-2}
        arXiv & \textcolor{Copyleft}{CC-BY-NC-SA-4.0} &  &   \\ \cline{1-2}
        PubMed & \textcolor{Copyleft}{CC-BY-NC-SA-4.0} &  &    \\ \cline{1-2}
        Deep-sequoia & CC-BY-NC-ND-4.0 &  &   \\ \cline{1-2} \cline{4-4}
  
        Midjourney Gen & CC-BY-NC-ND-4.0 &  & \multirow{6}{*}{Image}  \\ \cline{1-2}
        Flickr & \textcolor{Copyleft}{CC-BY-NC-SA-4.0} &  &   \\ \cline{1-2}
        StockSnap & \textcolor{Public}{CC0-1.0} &  &   \\ \cline{1-2}
        Wikimedia & \textcolor{Copyleft}{CC-BY-SA-4.0} &  &   \\ \cline{1-2}
        OpenClipart & \textcolor{Public}{CC0-1.0} &  &   \\ \cline{1-2} \cline{4-4}
        
        ccMixter & \textcolor{Permissive}{CC-BY-NC-4.0} & & \multirow{2}{*}{Voice}  \\ \cline{1-2}
        Jamendo & CC-BY-NC-ND-4.0 &  &   \\ \cline{1-2} \cline{4-4}
        
        Thingverse & \textcolor{Copyleft}{CC-BY-NC-SA-4.0} &  & 3D model  \\ \cline{1-2} \cline{4-4}
  
        Vimeo & CC-BY-NC-ND-4.0 &  & Video  \\ \hline
  
        Baize & \textcolor{Copyleft}{GPL-3.0} & \multirow{11}{*}{Model} & \multirow{4}{*}{Text Generation}   \\ \cline{1-2}
        BLOOM & \textcolor{Permissive}{BigScience-BLOOM-RAIL-1.0} & &   \\ \cline{1-2}
        Llama2 & \textcolor{Permissive}{Llama2 Community License} & &   \\ \cline{1-2}
        BigTranslate & \textcolor{Copyleft}{GPL-3.0} & &   \\ \cline{1-2} \cline{4-4}
  
        BERT & \textcolor{Permissive}{Apache-2.0} &  & Fill-Mask   \\ \cline{1-2} \cline{4-4}
  
        Stable Diffusion & \textcolor{Permissive}{CreativeML-OpenRAIL-M} & & Text to Image  \\ \cline{1-2} \cline{4-4}
  
        MaskFormer & \textcolor{Permissive}{CC-BY-NC-4.0} & & Image  \\ \cline{1-2}
        DETR & \textcolor{Permissive}{Apache-2.0} & & Segmentation \\ \cline{1-2} \cline{4-4}
  
        Whisper & \textcolor{Permissive}{MIT} & & Voice to Text  \\ \cline{1-2} \cline{4-4}
  
        X-Clip & \textcolor{Permissive}{MIT} & & Video to Text  \\ \cline{1-2} \cline{4-4}
  
        I2VGen-XL & CC-BY-NC-ND-4.0 & & Image to Video  \\ \hline
  
        \end{tabular}
        \footnotetext{}
  \end{table}
\end{comment}

%from HuggingFace\footnote{\url{https://huggingface.co/models}}
%We have encoded commonly used OSS, model, and free-content licenses and \textit{MG Licenses} into \textit{MG Analyzer} to illustrate their potential risks in model publishing scenarios. 
%\textit{MG Licenses} offer flexibility in encompassing nearly all licensing conditions provided by other model licenses.

%\section{Background and Related Work}
\label{sec:related}

\subsection{License Compliance Analysis}
The previous license compliance analysis studies primarily focus on OSS-licensed software~\cite{schoettle2019open, cui2023empirical}, and several successful tools, such as FOSSology~\cite{jaeger2017fossology} and Black Duck Software Composition Analysis~\cite{blackduck2024}, have been developed.
The main goal of these tools is to identify all open-source dependencies in software projects to evaluate associated license compliance risks, obligations, and attribution requirements.
Typically, the component dependencies and license information in a software project can be obtained through scanning and matching~\cite{ombredanne2020free}. 
This process involves gathering information from notices, headers, licenses, and other project files or attempting to match the code to determine its provenance.
Unfortunately, these strategies cannot be naturally extended to ML projects for the following reasons.

First, the components of ML projects, particularly models, have more intricate dependencies than code.
For example, knowledge can be transferred between models without explicitly copying weights~\cite{you2021workshop}.
Second, the OSS Bill of Materials standard, such as Software Package Data Exchange (SPDX)~\cite{spdx2024}, does not fully support common model licenses like OpenRAIL-M~\cite{contractor2022behavioral}, Llama 2~\cite{meta2024llama2}.
Third, non-standard licensing (model publishing under OSS or free-content licenses) is prevalent in ML projects~\cite{duan2024modelgo}, adding complexity to license analysis.
Furthermore, the crowd-sourced nature of ML components may also lead to over-permissive licenses~\cite{rajbahadur2021can}, distorting the analysis results.

For these reasons, license compliance issues in the ML field remain nearly unexplored. Rajbahadur \textit{et al.}~\cite{rajbahadur2021can} investigated license provenance issues in ML datasets and found instances of non-compliance between their licenses and the licenses of their data sources.
Building on these findings, Duan \textit{et al.}~\cite{duan2024modelgo} proposed a tool that analyzes conflicts directly based on the licenses of data sources and provided guidelines to minimize such conflicts.
However, their tool is limited to generating analysis reports and lacks the ability to visualize and exchange ML workflows, making it difficult to extend or integrate external resources from the web (e.g., linking to another workflow via URI).
Therefore, in this work, we propose an ontology describing ML workflows using RDF graphs (a visualized workflow is provided in Appendix~\ref{apdx:grapher}), making it both extensible and linkable, and then analyze license compliance through an automated reasoning engine.

\subsection{Model Licensing}
Today, we have many OSS licenses to accommodate diverse publishing scenarios~\cite{rosen2005open}. However, do they still function as intended in model publishing scenarios? The answer is no. While these licenses aim to govern the use and distribution of software, they lack definitions of ML concepts, which compromises their effectiveness (ref. Section~\ref{sec:license}).
Some model licenses (or agreements) are also emerging, such as Llama 2 and Gemma~\cite{gemma2024}. 
However, most of these licenses are specifically designed to govern certain models or their derivatives and are not as open as they claim to be~\cite{liesenfeld2024rethinking}. 
Meanwhile, Contractor \textit{et al.}~\cite{contractor2022behavioral} proposed OpenRAIL-M, a model license derived from Apache-2.0. 
While this license offers good clarity, it enforces use behavior restrictions that render it non-compliant with open-source licenses like GPL-3.0~\cite{greenbaum2016the}.
In addition, although OpenRAIL-M has many variations, its license terms are quite homogenized and lack the flexibility needed to accommodate different model publishing scenarios, such as non-commercial use, open sourcing, and restrictions on sharing outputs. 
Therefore, a significant number of developers have opted to publish their models using CC licenses, such as CC-BY-NC-4.0, as an alternative to prohibit commercial use of their models.
However, these licenses also face the issue of losing effectiveness in the context of model publishing.
Such unstandardized licensing practices can lead to increased compliance issues and pose potential legal hazards in ML projects~\cite{duan2024modelgo}. 
To promote standardized model publication, we propose a new set of model licenses that provide a wider range of licensing options.

%\section{MG Analyzer}
\label{sec:analyzer}

\begin{figure*}[t]
    \centering
    \includegraphics[width=\linewidth]{fig/framework.pdf}
    \caption{Overview of MG Analyzer. ("mg" is the prefix for our proposed vocabulary.)}
    \Description{}
    \label{fig:framework}
\end{figure*}

This section aims to introduce the specific design of MG Analyzer by exploring three questions: \emph{(i) How can we represent the workflows of ML projects? (ii) How do we establish a mapping from license text to reasonable rules? (iii) What types of license compliance issues can arise in ML projects, and how can we detect them?}
Before delving into the detailed design that answers these questions, we first provide an overview that serves as a roadmap for this section.
As illustrated in Figure~\ref{fig:framework}, the process of MG Analyzer is divided into three main parts: Construction, Reasoning, and Analysis.

In the \textbf{Construction Stage}, user-input workflow descriptions (written in Python) are converted into an RDF graph (saved as \emph{gen.ttl} in Turtle format) that contains the base information of the workflow. 
This conversion is achieved with the help of RDFLib~\cite{Krech_RDFLib_2023} and the \emph{MG Vocabulary}, which is part of our analyzer. 
RDFLib provides an API for writing RDF graphs, while the MG Vocabulary defines the specific semantics to represent the concepts and dependencies in ML projects.
Then, we apply reasoning rules (written in Notation3~\cite{arndt2023notation3}) for the complete workflow construction using the EYE reasoner~\cite{verborgh2015drawing}.
The reasoner concludes new properties that represent the input and output chains between components, followed by further reasoning to identify the \emph{compositional dependencies} among them (reflecting Question (i); see Section~\ref{sec:wf} for details).

The main tasks in \textbf{Reasoning Stage} involve concluding the \emph{definition dependencies} and \emph{rights-using} dependencies. 
This is achieved through two substeps.
First, a new property called \emph{ruling} is created to record the definition of the output work in relation to the input work within the context of licensing.
For example, if we merge GPL-licensed code into another software, the resulting work is considered a \emph{derivative} of the original work. 
This relationship, which we refer to as \emph{definition dependencies}, is crucial for determining the applicable license of the output work. 
We recursively identify such dependencies and ascertain the licenses of indeterminate works until all works in the workflow have a license.
Based on the RDF workflow graph with complete license assignments, we can execute the second step of reasoning, called \emph{request}, which infers \emph{rights-using} dependencies that represent the rights required for the work according to practical reuse methods (reflecting Question (ii); see Section~\ref{sec:rule} for details). 

So far, all necessary information for license analysis has been concluded before entering the final \textbf{Analysis Stage}. 
In this stage, MG Analyzer evaluates the validity of the base workflow information, checks for the satisfaction of rights granting, and assesses license compliance and conflicts. 
Entities of the class \emph{Report} are genreated to present these results in RDF format (reflecting Question (iii); see Section~\ref{sec:analysis} for details).

\subsection{ML Workflow Representation}
\label{sec:wf}
% 机器学习流程的描述(特别是用于许可分析)并不简单,原因有1)同时存在多种形式的组建和许可;2)组件间的依赖关系可能是隐含的和嵌套的;3)输出的定义是取决于许可的
The representation of ML workflows, particularly when considering license analysis scenarios, differs significantly from common software workflows for the following three reasons.

\ding{172} ML workflows often involve various components (e.g., code, datasets, images, model weights, services), each governed by licenses from different frameworks. 
Additionally, non-standard licensing practices are prevalent in current ML projects~\cite{rajbahadur2021can, duan2024modelgo}, for instance, C4AI Command R+ model~\cite{cohereforai2024c4ai} is licensed under a free-content license: CC-BY-NC-4.0. 
Therefore, the representation should be flexible enough to cover such situations.

\ding{173} The component dependencies in ML workflows may be implicit and nested. 
For instance, Openjourney~\cite{openjourney2024prompthero} is fine-tuned based on StableDiffusion~\cite{rombach2022high} model and the data generated by Midjourney~\cite{midjourney2024terms}. 
In this case, knowledge from Midjourney is transferred to Openjourney without explicit compositional inclusion. 
Therefore, the representation should consider the multifarious dependencies present within ML projects.

\ding{174} The components' dependencies are also defined by the components' practical licenses and the ways they are reused. 
A common case in the OSS field is that republishing Software as a Service (SaaS) is considered to convey a \emph{derivative} under AGPL-3.0 but has \emph{no definition} under GPL-3.0. 
Therefore, terms like \emph{derivative} and \emph{independent} should be contextualized within specific licenses, and our representation should be capable of reflecting such meanings.

Therefore, we propose the MG Vocabulary to describe the properties and classes within ML workflows.
For the flexibility issue \ding{172}, we use the following terms to abstract key concepts of ML workflows:

\textbf{Work}: Represents the components (e.g., models, datasets), each with a unique Type and Form. 
A work can have a license assigned through a Register License Action, or its license can be determined through rules applied in the Reasoning Stage (ref Figure~\ref{fig:framework}).

\textbf{Action}: Represents operations performed on a Work, including Modify, Train and Combine, etc. 
In practice, we broaden the definition of these operations to make the vocabulary adaptable to different types of works. (See Table~\ref{tab:action} for more details.)

\textbf{Work Type}: Includes software, dataset, model, and mixed-type. It is used to describe both the nature of the work and to identify the types of materials intended by a license. 
We utilize this information to detect mismatches between a work's type and its license.

\textbf{Work Form}: Divided into three subclasses: Raw, Binary, and Service to provide flexibility. 
For example, source code, model weights, and corpus fall under Raw; compiled programs are considered Binary; and SaaS or online chat LLMs are categorized as Service.
Additionally, three general terms are offered: raw-form, binary-form, and service-form, which can work in conjunction with the work type. 
This approach helps represent concepts that lack a formal designation, such as "a dataset published as a service".
We use mixed-form to represent the cases involving collections of works.

\textbf{LicenseInfo}: This contains the essential license information derived from conditions, including the license name, ID, intended types of works, whether it is copyleft or permissive, as well as granted and reserved rights, etc.
While the license name is sufficient to describe the base workflow, to enable reasoning, LicenseInfo should bind rules, which we will discuss in the next section.

At this stage, we can describe a base ML workflow, as illustrated\footnote{~The "mg" prefix is omitted and some properties are merged or filtered for presentation.} schematically in Figure~\ref{fig:wf} (The RDF graph can be referred to in \emph{gen.ttl} in Figure~\ref{fig:framework}).
In this base workflow, the derived input and output of each Action can be represented as blank nodes, serving as placeholders. 
These placeholders will be populated by reasoning the output yielded by the previous action and the input for the current action, respectively.

For dependency issues discussed in \ding{173} and \ding{174}, we identify three potential types of dependencies in an ML project: \textbf{compositional, definition, and rights-using}. These dependencies are visually represented by different colored dashed arrows in Figure~\ref{fig:wf}.
\emph{Compositional dependencies} are categorized into four types: \emph{Mixwork}, \emph{Subwork}, \emph{Auxwork}, and \emph{Provenance}, each representing the containment relationships between input and output works.
For example, when the output work includes the input work or a part of it, the input is considered the Mixwork of the output. 
This type of dependency is vital for license analysis because all actions performed on the output work, including any rights usage, will proliferate to the Mixwork components.
For instance, when fine-tuning an ensemble model~\cite{furlanello2018born}, the fine-tuning operation will cascade to all submodels. 
Consequently, the license terms related to fine-tuning for each submodel are triggered, meaning that any constraints or rights imposed on the submodels must be honored. 
Additionally, if a work includes Mixworks in different forms, such as code and weights, we need to generalize the output's form to raw-form to accommodate these variations.
A similar approach applies to the work type as well.
Subwork and Auxwork are used to track works that are utilized by other works in the workflow, such as training datasets or distilled models. 
The key distinction is that Subwork is intended to be published alongside the output, which necessitates additional license analysis related to republishing.
Provenance is specifically used for the Register License action to indicate that the output is simply the input itself, bound to a license. In such cases, any further proliferation of dependencies should cease.

The \emph{definition dependencies} represent the relationships between works based on the definitions established by their licenses. 
For example, if a new work is created by modifying GPL-licensed code, that modification is considered a \emph{derivative} of the original work.
These dependencies should be understood in the context of the original license and can extend to subsequent actions if the same conditions are activated again (e.g., a \emph{derivative} of a \emph{derivative} is likely still a \emph{derivative}).
These dependencies are the main factor in determining the applicable license and restrictions for the output work. 
For example, if the original work is under GPL-3.0, the republication of \emph{derivatives} must also apply the GPL-3.0 license.
Additionally, a work may have multiple definition dependencies in a complex workflow, and these dependencies should be simultaneously satisfied during license determination (if possible; otherwise, an error should be reported).
The corresponding implementation in the MG Analyzer is illustrated in the \emph{Reasoning Stage} of Figure~\ref{fig:framework}. 
New instance nodes called \emph{Ruling} are created to track the definition dependencies and triggered rules for each work, determining their applicable licenses in an alternating manner.

The \emph{rights-using dependencies} describe the rights that must be granted for actions performed on works. 
For example, when executing a Train action on a model, it requires the rights to \emph{use} and \emph{modify} (termed as \emph{Usage} in MG Vocabulary) from the model's license\footnote{~An example of such rights granting can be found in the Llama 2 Community License, which states, \emph{"You are granted a ... to use, reproduce, distribute, copy, create derivative works of, and make modifications to the Llama Materials."}.}.
Similarly, the rights-using dependencies should proliferate according to compositional dependencies.
For instance, the requirement to \emph{modify} a model extends to all its submodels.
In the MG Analyzer, we create new nodes called \emph{Request} to represent this dependency.
Additionally, it is insufficient to only check the granting rights; the reserved rights must also be verified. 
Depending on the clarity of the license text, some rights may either be explicitly granted or reserved.
Furthermore, certain license clauses can waive the requirement for specific rights. 
For instance, both GPL-3.0 and CC licenses include automatic relicensing clauses for downstream recipients, which eliminate the need for a \emph{sublicense} right.

By MG Vocabulary, we are able to describe complete ML workflows and represent the necessary dependencies for license analysis.
The \emph{compositional dependencies} are license-independent and can be reasoned from the base workflow. 
However, \emph{definition dependencies} and \emph{rights-using dependencies} are associated with specific rules expressed in natural language within each license.
To facilitate automated reasoning for these dependencies, the next step is to develop a viable method for encoding license terms into formal logic rules.

\begin{figure}[t]
    \centering
    \includegraphics[width=\linewidth]{fig/wf.pdf}
    \caption{A Typical ML Workflow Represented by MG Analyzer. Dashed arrows with different colors indicate properties related to three types of dependencies: \textcolor{Construct}{compositional}, \textcolor{Ruling}{definition}, and \textcolor{Request}{rights-using} dependencies.}
    \Description{}
    \label{fig:wf}
\end{figure}

\subsection{License Rule Encoding and Reasoning}
\label{sec:rule}

\begin{table}[t]
    \caption{Supported Actions in MG Analyzer Following Rule Alignment. The symbols $=$ and $\approx$ indicate that the Type/Form of output work and input work are the same and may differ, respectively. The corresponding \textcolor{Construct}{OSS}, \textcolor{Ruling}{Free-content}, and \textcolor{Request}{Model} license terms for each action are listed. }
    \tiny
    \label{tab:action}
    \begin{tabular}{|p{0.8cm}|c|c|p{3.6cm}|p{1.5cm}|}
    \hline
    \rowcolor[gray]{.8}
    Action & Type & Form & Composition & Terms \\ \hline

    Copy & $=$ & $=$ & Output and input are exactly \textbf{same}. & Copy Duplicate \\ \hline

    Combine & $\approx$ & $\approx$ & \textbf{Entire} input included in output. & \textcolor{Construct}{Link} \textcolor{Construct}{Aggregate} \textcolor{Request}{MoE} \textcolor{Ruling}{Arrange} \textcolor{Ruling}{Collect} \\ \hline

    Modify & $=$ & $=$ & Output includes a \textbf{significant} portion of input and \textbf{can be reverted}. & Modify \textcolor{Request}{Fine-tune}   \\ \hline

    Amalgamate & $=$ & $=$ & Output includes portions of input but \textbf{cannot be reverted}. & Modify \textcolor{Ruling}{Remix} \textcolor{Request}{Fusion} \\ \hline

    Train & $=$ & $=$ & Output has the \textbf{same structure} as input and may contain a \textbf{negligible} portion of it. & \textcolor{Ruling}{Alter Adapt} \textcolor{Request}{Train} \\ \hline

    Generate & $\approx$ & $\approx$ & Output does not contain any portion of input and may \textbf{perform differently} from it. & Output \textcolor{Request}{Generate} \newline \textcolor{Request}{Synthetic} \\ \hline

    Distill & $=$ & $=$ & Output does not contain any portion of input but \textbf{performs similarly} to it. & \textcolor{Request}{Distill Transfer} \newline \textcolor{Ruling}{Extract} \\ \hline

    Embed & $=$ & $=$ & Output does not contain any portion of input, but there has a \textbf{mapping} that converts input to output. & \textcolor{Ruling}{Translate Transform} \\ \hline

    Publish & $=$ & $\approx$ & Output is the \textbf{same} as the input but may have a \textbf{different form}.  & \textcolor{Construct}{Redistribute} \textcolor{Ruling}{Perform Display Disseminate} \\ \hline
    \end{tabular}
\end{table} 

Typically, licenses are designed to govern the use and distribution of specific types of works. 
For example, GPL-3.0 is tailored for source code and object code, while CC licenses focus on literary, musical, and artistic works. 
As a result, it is challenging to map their rules within a unified framework for logical reasoning.
Meanwhile, many ML projects actually incorporate non-standard licensing components, as mentioned in Section~\ref{sec:intro}.
If we consider these claimed licenses to be invalid, then they would not pose any license compliance issues.
However, the validity of these licenses depends on specific cases and the dispute resolution process by the jurisdictional courts in accordance with the applicable laws in different regions.
As a license analyzer, we aim to maximize the detection of all potential legal risks under various interpretations, rather than merely granting a green light with low confidence.
To this end, we perform three generalizations to encode license rules.

The first generalization is called \emph{fuzz form matching}. 
We broaden the definitions related to a work's form to encompass its general form.
For instance, we expand the license terms of GPL-3.0 concerning source code to include all forms of work in the Raw categories, such as model weights and corpus. 
In this way, we can extend the scope of interpretation of GPL-3.0 to cover models and datasets.

The second generalization is called \emph{composition-based rule alignment}, which aims to resolve the pervasive ambiguities across different licensing frameworks. 
This ambiguity often arises in non-standard scenarios, for example, when licensing a model under GPL-3.0, it may be unclear whether \emph{Model Aggregation} (a technology used in federated learning~\cite{li2021model}) triggers the "Aggregate" clause in GPL-3.0.
Therefore, we propose a composition-based method to align these rules. 
Specifically, we generalize the concept of action to represent the compositional relationships between input and output. 
For instance, the action \emph{Combine} signifies that the input work has been entirely included in the output work without modification. 
This action corresponds to terms such as "Link" and "Aggregate" in software licenses, "Collection" in CC licenses, and "MoE" in model licenses.
In the case of \emph{Model Aggregation}, which produces an output that contains parts of the input and is difficult to separate, it is not considered a \emph{Combine}. Consequently, according to our rule alignment, it will not activate the "Aggregate" clause in GPL-3.0.

The complete rule alignment method utilized in the MG Analyzer can be found in Table~\ref{tab:action}.
It is worth mentioning that the meanings of these actions have been broadened and may differ from their original definitions. 
In some cases, multiple actions may align with the same license terms. 
For instance, the license term "Modify" can align with both the actions \emph{Modify} and \emph{Amalgamate}, as such licenses do not distinguish the extent of changes made or whether those changes can be reverted.

The final aspect is \emph{applicable term generalization}, where we encode the triggering conditions of a license term into the following properties: range of input work forms, range of output work forms, and types of actions.
Figure~\ref{fig:mgrule} illustrates an example of the GPL-3.0 derivatives rule\footnote{~The original GPL-3.0 license text reads: \emph{"You may convey a work based on the Program ... in the form of source code ... provided that you also meet all of these conditions: ... stating that you modified it ... it is released under this License ... keep intact all notices ... license the entire work, as a whole, under this License ..."} .}, which represents the Combine action applied to input works in code format, resulting in a code output.
This action triggers the "derivative" clause in GPL-3.0, indicating that the license of the output work must be compatible with GPL-3.0 (e.g., APGL-3.0). Additionally, five restrictions must apply to the output work if it is to be republished, as dictated by this \emph{definition dependency}.
This rule does not include any \emph{Use Restrictions}, which apply to output works regardless of whether they are republished. 
We found this subtle distinction to be crucial in ML license analysis, as most OSS license terms are triggered by distribution, and their definitions of "distribution" typically exclude publishing as a service. 
However, in the case of models, the common deployment method is through a web interface,  leading to many OSS license restrictions being circumvented in such scenarios (ref Section~\ref{sec:case}).

Furthermore, multiple rules may lead to the same output definition, and we provide the option of \emph{fuzz form matching} by enabling \emph{fuzz rules} to enhance the interpretive capabilities of the MG Analyzer.
It is worth mentioning that the reasoning results behind the restrictions derived from these rules require further analysis for validation.
Taking the rule in Figure~\ref{fig:mgrule} as an example,  the presence of additional discrimination terms in the final work that violate GNU freedom~\cite{greenbaum2016the} dictates whether errors related to GNU freedom conflicts should be reported.
We present snippets of our encoded rules written in Turtle format in Appendix~\ref{apdx:rules}.
The list of supported licenses, whose terms have been encoded in MG Analyzer, is shown in Table~\ref{tab:licenses}, covering nearly all top-ranking licenses for published models on HuggingFace\footnote{~\url{https://huggingface.co/models}}, a popular model publishing platform.

\begin{figure}[t]
    \centering
    \includegraphics[]{fig/mgrule.pdf}
    \caption{The Example of a Generalized GPL-3.0 Derivatives Rule in MG Analyzer.}
    \Description{}
    \label{fig:mgrule}
\end{figure}

With MG Vocabulary, the ML workflow with compositional dependencies, and encoded license rules, we can reason to derive the definition and rights-using dependencies, thereby determining the applicable licenses for intermediate works. 
The license determination in the MG Analyzer follows an incremental and minimal noncompliance strategy, where the new license only applies to the incremental parts of the work without affecting the original work (a common practice in licensing). 
Furthermore, to avoid introducing additional compliance issues during analysis, we use \emph{Unlicense} as the default license when applicable.
However, an exception arises with the license proliferation clauses found in copyleft licenses, such as GPL-3.0 and OSL-3.0, which require that the entire new work be licensed under the same terms. 
Our analyzer incorporates reasoning logic to identify applicable licensing solutions (a snippet of logic can be found in Appendix~\ref{apdx:rules}), but unresolved conflicts may occur if multiple copyleft clauses are triggered. 
In such cases, the MG Analyzer will select one of these copyleft licenses and report an error during the \emph{Analysis Stage}.

% In addition, some ML concepts have no definition in other licensing framework but The latter has clause toward 

%more abstraction are need to perform in encoding license rules.

\subsection{Compliance Analysis}
\label{sec:analysis}

\begin{table}[tp]
    \caption{List of \textcolor{Ruling}{Notices}, \textcolor{Construct}{Warnings}, and \textcolor{Request}{Errors} Reported by MG Analyzer. The triggered work is denoted as \emph{?work}. }
    \scriptsize
    \label{tab:report}
    \begin{tabular}{|c|p{2.25cm}|p{4.6cm}|}
    \hline
    \rowcolor[gray]{.8}
    Code & Report Type & Report Content \\ \hline
    
    N1 & \textcolor{Ruling}{Include License} & The original license file from \emph{?work} should be retained.  \\ \hline

    N2 & \textcolor{Ruling}{Include Notice} & The notices (e.g., attribution, copyright, patent, trademark) from \emph{?work} should be retained. \\ \hline

    N3 &\textcolor{Ruling}{State Changes} & A notice stating the modifications made to \emph{?work} should be provided. \\ \hline

    N4 &\textcolor{Ruling}{ImpACT Reports} & You need to complete a Derivative Impact Report. \\ \hline

    W1 &\textcolor{Construct}{License Type Mismatch} & Non-standard licensing of \emph{?work}. \\ \hline
    
    W2 &\textcolor{Construct}{Revocable License} & The license of \emph{?work} is revocable. \\ \hline
    
    W3 &\textcolor{Construct}{Possibly Revocable License} & The revocability of the license of \emph{?work} is not claimed. \\ \hline

    W4 &\textcolor{Construct}{Right Not Granted} & The required right is not explicitly granted by \emph{?work}. \\ \hline

    W5 &\textcolor{Construct}{Disclose Source Code} & This work should disclose its source code. \\ \hline

    W6 &\textcolor{Construct}{Disclose Unmodified Code} & The unmodified source code of \emph{?work} should be disclosed. \\ \hline

    W7 &\textcolor{Construct}{Use Behavior} & The use of this work must comply with the usage behavior restrictions of \emph{?work}. \\ \hline

    W8 &\textcolor{Construct}{Runtime Control} & There is a runtime restriction clause in \emph{?work} (e.g., forced updates). \\ \hline

    E1 &\textcolor{Request}{Wrong Work Type or Form} & The type of \emph{?work} is inconsistent with its form. \\ \hline

    E2 &\textcolor{Request}{Right Reserved} & The required right is reserved by the license of \emph{?work}. \\ \hline
    
    E3 &\textcolor{Request}{Not Allowed to Share} & Redistribution of this work is prohibited. \\ \hline

    E4 &\textcolor{Request}{Not Allowed to Sublicense} & Sublicensing of \emph{?work} is prohibited. \\ \hline

    E5 &\textcolor{Request}{Non-Commercial Use} & Commercial use of \emph{?work} is prohibited. \\ \hline

    E6 &\textcolor{Request}{Cannot Be Relicensed} & The license of this work is invalid because \emph{?work} cannot be relicensed, or relicensing is prohibited. \\ \hline

    E7 &\textcolor{Request}{GNU Freedom Conflict} & The additional terms applied in this work may violate the GNU freedom clauses of \emph{?work}. \\ \hline

    E8 &\textcolor{Request}{CC Freedom Conflict} & The additional terms applied in this work may violate the CC freedom clauses of \emph{?work}. \\ \hline

    E9 &\textcolor{Request}{Llama 2/3 Exclusive} & Using Llama 2/3's output in non-Llama 2/3 derivatives is prohibited. \\ \hline

    E10 &\textcolor{Request}{Exclusive License} & The additional terms applied in this work are prohibited by the license of \emph{?work}. \\ \hline

    \end{tabular}
\end{table}


At this stage, we establish all necessary dependency properties through automated reasoning to enable compliant license analysis. 
The analysis rules are designed to assess and report the validity of the base workflow information, the fulfillment of granted rights, work restrictions, and overall license compliance. 
In addition, the Publish action should be invoked to signify the completion of the workflow, along with an assigned public manner and work form. 
MG Analyzer considers three republication scenarios: internal, share, and sell, each of which typically involves different terms and conditions in the licenses.
For instance, if we publish the final work for sale, the related licenses should grant rights for redistribution, sublicensing, and commercial use. 

Appendix~\ref{apdx:rules} presents the logic code for rights granting analysis, and the full list of reported notices, warnings, and errors is shown in Table~\ref{tab:report}.
Due to its staged logic rule reasoning design, MG Analyzer has considerable extensibility, enabling the incorporation of additional licenses and analysis targets, provided that it can reason based on our proposed dependencies information.
Beyond compliance analysis, our vision is to promote ML workflow supply chain management and improve FAIRness~\cite{wilkinson2016fair} in model publishing. 
We position our vocabulary and tool as a first step toward Linked Open Model Production Data~\cite{bizer2008linked}.



%\section{Experiments}
\label{sec:case}
This section seeks to answer a key question: \emph{Should I continue using traditional OSS and free-content licenses to publish my model, and what are the associated risks?}
To explore this question, we evaluate commonly adopted licenses in the context of model publishing by MG Analyzer to assess whether they are still effective as intended.
The example workflow involves combining two models and publishing them as a service, as shown in Figure~\ref{fig:case1}.
Here, we consider two scenarios: 1) publishing the combined model as a service; 2) publishing the data generated from the combined model.
Models A and B are non-binding works that correspond to the settings in Table~\ref{tab:result}, with their respective analysis results also presented in the table.
Model C is an intermediate work created by combining A and B, and Data D is the generated output from C. 
Work E involves republishing C as a service with the intent to sell, while Work F involves republishing D as a literary form, also with the intent to sell.
To be more convincing, we use real-world models and their respective licenses for demonstration.

\begin{figure}[t]
    \centering
    \includegraphics{fig/case1.pdf}
    \caption{Example Workflow: Combine Models, Then Publish.}
    \Description{}
    \label{fig:case1}
\end{figure}

\begin{table}[]
    \caption{Settings and Analysis Results.}
    \footnotesize
    \begin{tabular}{|ll|}
    \hline \rowcolor[gray]{.8}
    \multicolumn{1}{|l|}{Work: License} & Work: Report code. \\ \hline
    \multicolumn{1}{|p{3.2cm}|}{(i) C: AGPL-3.0. D: Unlicense. \newline (ii) D: Unlicense } & \multicolumn{1}{p{4.5cm}|}{ (i) E: N1N2N3$\times$2; \textbf{W5}$\times$2; W1$\times$3. F: W1$\times$3. \newline (ii) E: W1$\times$2. F: W1.} \\ \hline
    \multicolumn{2}{|p{8cm}|}{\textbf{OSS License Setting:} \newline
    (i) A$\leftarrow$PhoBERT~\cite{nguyen2020phobert}: AGPL-3.0. B$\leftarrow$CKIP-Transformers~\cite{lin2022hantrans}: GPL-3.0. \newline
    (ii) A$\leftarrow$PhoBERT~\cite{nguyen2020phobert}: AGPL-3.0. B$\leftarrow$\emph{None}.
    }\\ \hline \hline

    \multicolumn{1}{|p{3.2cm}|}{C: Unlicense. D: Unlicense.} & E: W1$\times$2, \textbf{E2}$\times$2. F: W1$\times$2 . \\ \hline
    \multicolumn{2}{|p{8cm}|}{\textbf{Free-content License Setting:}\newline
        A$\leftarrow$ MPT-Chat~\cite{MosaicML2023Introducing}: CC-BY-NC-SA-4.0.  B$\leftarrow$Command R+~\cite{cohereforai2024c4ai}: CC-BY-NC-4.0.
    } \\ \hline \hline

    \multicolumn{1}{|p{3.2cm}|}{(iii) D: Unlicense. \newline (iv) D: Unlicense.} & \multicolumn{1}{p{4.5cm}|}{ (iii) E: N1; N2: \textbf{W5} . F: \emph{None}. \newline (iv) E: N1; N2; W2$\times$2; \textbf{E2}; \textbf{E5}. F: W2, \textbf{E5}.} \\ \hline
    \multicolumn{2}{|p{8cm}|}{\textbf{Model License Setting:}\newline
        (iii) A$\leftarrow$ MG-BY-OS. B$\leftarrow$\emph{None}. (iv) A$\leftarrow$ MG-BY-NC. B$\leftarrow$\emph{None}.
    } \\ \hline
    \end{tabular}
    \label{tab:result}
\end{table}

First, we evaluate two OSS licenses: AGPL-3.0 and GPL-3.0, which are considered enforceable open-source licenses with copyleft clauses.
In the setting (i), Work E triggers two \emph{Disclose Source Code} warnings (code W5, refer to Table~\ref{tab:report} for code definitions), and AGPL-3.0 successfully proliferates to Work C.
However, with a small adjustment, we can circumvent these clauses by republishing the generated content rather than providing the model as a service. 
As demonstrated by Work D, there is no W5 warning, and the content is licensed under the Unlicense.
Furthermore, the condition in AGPL-3.0 that triggers the disclose code clauses related to remote access is \emph{you modify the Program}, which means you can directly republish copies as a service to circumvent this clause.
As reflected in setting (ii), there are no more W5 warnings, only two warnings related to the non-standard licensing remain.





%%
%% The acknowledgments section is defined using the "acks" environment
%% (and NOT an unnumbered section). This ensures the proper
%% identification of the section in the article metadata, and the
%% consistent spelling of the heading.
% \begin{acks}
% ack.
% \end{acks}

%%
%% The next two lines define the bibliography style to be used, and
%% the bibliography file.
\bibliographystyle{ACM-Reference-Format}
\bibliography{REF}


%%
%% If your work has an appendix, this is the place to put it.
\appendix
apd

\end{document}
\endinput
%%
%% End of file `sample-sigconf.tex'.
 