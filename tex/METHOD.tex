\section{MG Analyzer}
\label{sec:analyzer}

\begin{figure*}[t]
    \centering
    \includegraphics[width=\linewidth]{fig/framework.pdf}
    \caption{Overview of MG Analyzer ("mg" is the prefix of our proposed vocabulary).}
    \Description{}
    \label{fig:framework}
\end{figure*}

This section aims to introduce the specific design of MG Analyzer by exploring three questions: \emph{(i) How can we represent the workflows of ML projects? (ii) How do we establish a mapping from license text to reasonable rules? (iii) What types of license compliance issues can arise in ML projects, and how can we detect them?}
Before delving into the detailed design that answers these questions, we first provide an overview that serves as a roadmap for this section.
As illustrated in Figure~\ref{fig:framework}, the process of MG Analyzer is divided into three main parts: Construction, Reasoning, and Analysis.

In the \textbf{Construction Stage}, user-input workflow descriptions (written in Python) are converted into an RDF graph (saved as \emph{gen.ttl} in Turtle format) that contains the base information of the workflow. 
This conversion is achieved with the help of RDFLib~\cite{Krech_RDFLib_2023} and the MG vocabulary, which is part of our analyzer. 
RDFLib provides an API for writing RDF graphs, while the MG vocabulary defines the specific semantics to represent the concepts and dependencies in ML projects.
Then, we apply reasoning rules (written in Notation3~\cite{arndt2023notation3}) for the complete workflow construction using the EYE reasoner~\cite{verborgh2015drawing}.
The reasoner concludes new properties that represent the input and output chains between components, enabling further reasoning about the \emph{compositional dependencies} among them (reflecting Question (i); see Section~\ref{sec:wf} for details).

The main tasks in \textbf{Reasoning Stage} involve concluding the \emph{definition dependencies} and \emph{rights-using} dependencies. 
This is achieved through two substeps.
First, a new property called \emph{ruling} is created to record the definition of the output work in relation to the input work within the context of licensing.
For example, if we merge GPL-licensed code into another software, the resulting work is considered a \emph{derivative} of the original work. 
This relationship, which we refer to as \emph{definition dependencies}, is crucial for determining the compliant license of the output work. 
We recursively identify such dependencies and ascertain the licenses of indeterminate works until all works in the workflow have a license.
Based on the RDF workflow graph with complete license assignments, we can execute the second step of reasoning, termed \emph{request}, which records \emph{rights-using} dependencies that represent the rights required for the work according to practical reuse methods (reflecting Question (ii); see Section~\ref{sec:rule} for details). 

So far, all necessary information for license analysis has been concluded before entering the final \textbf{Analysis Stage}. 
In this stage, MG Analyzer evaluates the correctness of the base workflow information, checks for the satisfaction of rights granting, and assesses license compliance and conflicts. 
\emph{Reports} are genreated to present these results in the workflow graph (reflecting Question (iii); see Section~\ref{sec:analysis} for details).


\subsection{ML Workflow Representation}
\label{sec:wf}
% 机器学习流程的描述(特别是用于许可分析)并不简单,原因有1)同时存在多种形式的组建和许可;2)组件间的依赖关系可能是隐含的和嵌套的;3)输出的定义是取决于许可的
The representation of ML workflows, particularly when considering license analysis scenarios, differs significantly from common software workflows for the following three reasons.
\ding{172} ML workflows often involve various components (e.g., code, datasets, images, model weights, services), each governed by licenses from different frameworks. 
Additionally, non-standard licensing practices are prevalent in current ML projects, for instance, the C4AI Command R+ model~\cite{cohereforai2024c4ai} is licensed under a free-content license: CC BY-NC-4.0. 
Therefore, the representation should be flexible enough to cover such situations.

\ding{173} The component dependencies in ML workflows may be implicit and nested. 
For instance, Openjourney~\cite{openjourney2024prompthero} is fine-tuned based on the StableDiffusion~\cite{rombach2022high} model and the data generated by Midjourney~\cite{midjourney2024terms}. 
In this case, knowledge from Midjourney is transferred to Openjourney without explicit compositional inclusion. 
Therefore, the representation should consider the multifarious dependencies present within ML projects.

\ding{174} The components' dependencies are also defined by the components' practical licenses and the ways they are reused. 
A common case in the OSS field is that republishing Software as a Service (SaaS) is considered to convey a \emph{derivative} under AGPL-3.0 but has \emph{no definition} under GPL-3.0. 
Therefore, terms like \emph{derivative} and \emph{independent} should be contextualized within specific licenses, and our representation should be capable of reflecting such meanings.

Therefore, we propose the MG Vocabulary to describe the properties and classes in an ML workflow.
For the flexibility issue \ding{172}, we use the following terms to abstract key concepts in the workflow:

\textbf{Work}: Represents the components (e.g., models, datasets), each with a unique Type and Form. 
A work can have a license assigned through a Register License action, or its license can be determined through rules applied in the Reasoning Stage (see Figure~\ref{fig:framework}).

\textbf{Action}: Represents operations performed on a Work, including Modify, Train and Combine, etc. 
In practice, we broaden the definition of these operations to make the vocabulary adaptable to different types of works. (See ref~xxxxx for more details.)

\textbf{Work Type}: Includes software, dataset, model, and mixed-type. It is used to describe both the nature of the work and to identify the types of materials intended by a license. 
We use this information to detect any mismatch between the work type and the license.

\textbf{Work Form}: Divided into three subclasses: Raw, Binary, and Service to provide flexibility. 
For example, source code, model weights, and corpus fall under Raw; compiled programs are considered Binary; and services like SaaS or online chat LLMs are categorized as Service.
Additionally, three general terms are offered: raw-form, binary-form, and service-form, which can work in conjunction with the work type. 
This approach helps represent situations that lack a formal ontology, such as "a dataset published as a service".
We use mixed-form to represent the cases involving collections of works.

\textbf{LicenseInfo}: This contains the essential license information derived from conditions, including the license name, ID, intended types of works, whether it is copyleft or permissive, as well as granted and reserved rights, etc.
While the license name is sufficient to describe the base workflow, to enable reasoning, LicenseInfo should bind rules, which we will discuss in the next section.

At this stage, we can describe a base ML workflow, as illustrated\footnote{The "mg" prefix is omitted, and some properties are merged or filtered for presentation.} schematically in Figure~\ref{fig:wf} (The RDF graph can be referred to in \emph{gen.ttl} in Figure~\ref{fig:framework}).
In this base workflow, the derived input and output of each Action can be represented as blank nodes, serving as placeholders. 
These placeholders will be populated by reasoning the output yielded by the previous action and the input for the current action, respectively.

For dependency issues discussed in \ding{173} and \ding{174}, we identify three potential types of dependencies in an ML project: \textbf{compositional, definition, and rights-using}. These dependencies are visually represented by different colored dashed arrows in Figure~\ref{fig:wf}.
\emph{Compositional dependencies} are categorized into four types: Mixwork, Subwork, Auxwork, and Provenance, each representing the containment relationships between input and output works.
For example, when the output work includes the input work or a part of it, the input is considered the Mixwork of the output. 
This type of dependency is vital for license analysis because all actions performed on the output work, including any rights usage, will proliferate to the Mixwork components.
For instance, when fine-tuning a MoE model, the fine-tuning operation will cascade to all submodels. 
Consequently, the license terms related to fine-tuning for each submodel are triggered, meaning that any constraints or rights imposed on the submodels must be honored. 
Additionally, if a work includes Mixworks in different forms, such as code and weights, we need to generalize the output's form to raw-form to accommodate these variations.
A similar approach applies to the work type as well.
Subwork and Auxwork are used to track works that are utilized by other works in the workflow, such as training datasets or distilled models. 
The key distinction is that Subwork is intended to be published alongside the output, which necessitates additional license analysis related to republishing.
Provenance is specifically used for the Register License action to indicate that the output is simply the input itself, bound to a license. In such cases, any further proliferation of dependencies should cease.

The \emph{definition dependencies} represent the relationships between works based on the definitions established by their licenses. 
For example, if a new work is created by modifying GPL-licensed code, that modification is considered a \emph{derivative} of the original work.
These dependencies should be understood in the context of the original license and can extend to subsequent actions if the same conditions are activated again.
These dependencies are the main factor in determining the applicable license and restrictions for the output work. 
For example, under GPL-3.0, the republication of \emph{derivatives} must apply the same license.
Additionally, a work may have multiple definition dependencies in a complex workflow, and these dependencies should be simultaneously satisfied (if possible; otherwise, an error should be reported) during license determination.
The corresponding implementation in the MG Analyzer is illustrated in the Reasoning Stage of Figure~\ref{fig:framework}. 
New instance nodes called \emph{Ruling} are created to track the definition dependencies and triggered rules for each work, determining their applicable licenses in an alternating manner.

The \emph{rights-using dependencies} describe the rights that must be granted for actions performed on works. 
For example, when executing a training action on a model, it requires the rights to \emph{use} and \emph{modify} (termed as Usage in MG vocabulary) from the model's license\footnote{An example of such rights granting can be found in the Llama2 Community License~\cite{meta2024llama2}, which states, "You are granted a ... to use, reproduce, distribute, copy, create derivative works of, and make modifications to the Llama Materials.}.
Similarly, the rights-using dependencies should proliferate according to compositional dependencies.
For instance, the requirement to \emph{modify} a model extends to all its submodels.
In the MG Analyzer, we create new nodes called \emph{Request} to represent this dependency.
Additionally, it is insufficient to only check the granting rights; the reserved rights must also be verified. 
Depending on the clarity of the license text, some rights may either be explicitly granted or reserved.
Furthermore, certain license clauses can waive the requirement for specific rights. 
For instance, both GPL-3.0 and CC licenses include automatic relicensing clauses for downstream recipients, which eliminate the need for a \emph{sublicense} right.

By MG Vocabulary, we are able to describe complete ML workflows and represent the necessary dependencies for license analysis.
The \emph{compositional dependencies} are license-independent and can be reasoned from the base workflow. 
However, \emph{definition dependencies} and \emph{rights-using dependencies} are associated with specific rules expressed in natural language within each license.
To facilitate automated reasoning for these dependencies, the next step is to develop a viable method for encoding license terms into formal logic rules.

\begin{figure}[t]
    \centering
    \includegraphics[width=\linewidth]{fig/wf.pdf}
    \caption{A Typical ML Workflow Represented by MG Analyzer. Dashed arrows with different colors indicate the properties related to three kinds of dependencies: \textcolor{Construct}{compositional}, \textcolor{Ruling}{definition}, and \textcolor{Request}{rights-using} dependencies.}
    \Description{}
    \label{fig:wf}
\end{figure}

\subsection{License Rule Encoding and Reasoning}
\label{sec:rule}
Typically, licenses are designed to govern the use and distribution of specific types of works. 
For example, the GPL-3.0 is tailored for software and object code, while CC licenses focus on literary, musical, and visual works. 
As a result, it is challenging to map their rules within a unified framework for logical reasoning.
Meanwhile, many ML projects actually incorporate non-standard licensing components, as mentioned in Section~\ref{sec:intro}.
If we consider these claimed licenses to be invalid, then they would not pose any license compliance issues.
However, the validity of these licenses depends on specific cases and the dispute resolution process by the jurisdictional courts in accordance with the applicable laws in different regions.
As a license analyzer, we aim to maximize the detection of all potential legal risks under various interpretations,rather than merely granting a green light with low confidence.
To this end, we perform three generalizations to encode license rules.

The first generalization is called \emph{fuzz form matching}. 
We broaden the definitions related to a work's form to encompass its general form.
For instance, we expand the license terms of GPL-3.0 concerning source code to include all forms of work in the Raw categories, such as model weights and corpus. 
In this way, we can extend the scope of interpretation of GPL-3.0 to cover models and datasets.

The second generalization is called \emph{composition-based rule alignment}, which aims to resolve the pervasive ambiguities across different licensing frameworks. 
This ambiguity often arises in non-standard scenarios, for example, when licensing a model under GPL-3.0, it may be unclear whether \emph{model aggregation} (a technology used in federated learning~\cite{li2021model}) triggers the \emph{aggregate} clause in GPL-3.0.
Therefore, we propose a composition-based method to align these rules. 
Specifically, we generalize the concept of action to represent the compositional relationships between input and output. 
For instance, the action \emph{Combine} signifies that the input work has been entirely included in the output work without modification. 
This action corresponds to terms such as \emph{Link} and \emph{Aggregate} in software licenses, \emph{Collection} in CC licenses, and \emph{MoE} in model licenses.
In the case of \emph{model aggregation}, which produces an output that contains parts of the input works and is difficult to separate, it is not considered a \emph{Combine}. Consequently, according to our rule alignment, it will not activate the \emph{Aggregate} clause in GPL-3.0.
The complete rule alignment method utilized in the MG Analyzer can be found in Table~\ref{???}.


\begin{table}[]
    \caption{List of Supported Actions in MG Analyzer Following Rule Alignment. \textcolor{Construct}{OSS}, \textcolor{Ruling}{Free-content}, \textcolor{Request}{Model}}
    \tiny
    \begin{tabular}{|l|c|c|p{3.5cm}|p{1.5cm}|}
    \hline
    \rowcolor[gray]{.8}
    Action & Type & Form & Composition & Terms \\ \hline

    Copy & $=$ & $=$ & Input and output are exactly equal. & Copy Duplicate \\ \hline
    Combine & $\approx$ & $\approx$ & Entire input included in output. & \textcolor{Construct}{Link} \textcolor{Construct}{Aggregate} \textcolor{Request}{MoE} \textcolor{Ruling}{Arrange} \textcolor{Ruling}{Collection} \\ \hline

    Modify & $=$ & $=$ & Output includes a significant portion of the input and can be reverted. &  \\ \hline
    Amalgamate & $=$ & $=$ & Output includes portions of the input but cannot be reverted. &               \\ \hline
    Train & $=$ & $=$ & Output shares a similar structure with the input and may contain a negligible portion of it. &               \\ \hline
    Generate & $\approx$ & $\approx$ & &               \\ \hline
    Distill &      &      &             &               \\ \hline
    Publish &      &      &             &               \\ \hline
    \end{tabular}
\end{table}



% In addition, some ML concepts have no definition in other licensing framework but The latter has clause toward 

%more abstraction are need to perform in encoding license rules.

\subsection{Compliance Analysis}
\label{sec:analysis}

\begin{comment}
This section is organized around three key questions in ML license analysis: \emph{(i) How to determine the applicable conditions in licenses for certain model reuse mechanisms? (ii) How to capture the dependency structure of a ML project? (iii) What types of non-compliance exist in ML projects and how to assess them?}
We will present our solutions to these questions in the following sections.

\subsection{Taxonomy for ML License Analysis}
Determining the corresponding conditions in licenses is a challenging task for ML projects due to the conceptual ambiguities in existing licensing language and the disorganization in current ML licensing practices.
For example, license like CC-BY-ND prohibits the sharing of derivatives of licensed materials.
However, its definition of making derivatives is unclear in the ML domain.
For instance, should embeddings of a corpus be considered a derivative work upon that corpus?
Unfortunately, even though Creative Commons provides a flow chart to illustrate the trigger conditions of CC licenses in the context of AI activity~\cite{creative2023artificial}, it raises another question: \textit{Is the output considered protectable copyright subject matter?}
The answer depends on how the embedding activity is interpreted, for example, considering it as a translation of the original work can trigger the CC licenses.

MDL advocates the use of a \textit{Top Sheet} to delineate what ML activities are allowed with data~\cite{benjamin2019towards}, but this proposal is rarely implemented in practice (life would be easier if it were widely accepted). 
Making things more complex, some projects release their models under free content licenses, like LayoutLMv3 model~\cite{huang2022layoutlmv3}, which is licensed under CC-BY-NC-SA-4.0. 
This disorganization makes it unclear what kinds of ML activities can trigger licenses conditions in different contexts.
An ideal and elegant solution would be to encourage licensors to make context-appropriate adaptations in their license agreements or terms of use to clarify the granted rights related to ML activities. 
However, some ML components may be composed of prior works that are shared under copyleft license templates, which may disallow such relicensing of their derivatives to a new license.
Therefore, it is necessary to establish practical rules to bridge AI activities and existing licensing language.

To address the above challenge, we propose a new taxonomy that categorizes all AI activities into four categories based on the forms of their results. 
There are four categories of AI activities following our taxonomy: Combination, Amalgamation, Distillation, and Generation, which are defined by four forms of their results, respectively: 1) Combination with strong separation; 2) Combination with weak separation; 3) Derivatives from concepts; and 4) Derivatives from data.
Correspondingly, we can also categorize the usage behaviors in licensing language into these four categories based on their outcome forms.

We leverage Figure~\ref{fig:tax} to illustrate this idea.
The left side consists of a list of AI activities, many of which pertain to model reusing methods, categorized based on the forms of their results.
The middle part is our taxonomy that can classify these AI activities.
Following this rule, we can also identify the corresponding terms in natural language license text shown on the right side.
For example, Mixture of Experts (MoE) leverages a gating network to ensemble a batch of weak learners~\cite{jacobs1991adaptive}, which leads to a combination with strong separation and aligns with licensing terms like link, portion, collection, etc.
Unlike combination, the results of amalgamation are difficult (or impossible) to separate, corresponding to AI activities such as modification, fine-tuning, model fusion, etc~\footnote{Whether embeddings constitute a combination with weak separation depends on the specific case. In ModelGo, we classify embeddings as amalgamation if they are created under a content license that treats translation as a form of modification.}. 
These unrecoverable revision of original works are corresponding license text like adapt, alter, remix, etc.
On the other hand, distillation and generation are derivatives of original works, which means the results will not contain any portion of the original works. 
These two AI activities are mostly defined in AI model licenses but are not covered by traditional OSS licenses and free content licenses.

By now, we can ascertain the suitable permissions, limitations, and obligations for each AI activity based on the license language, even when the license type is not an exact match.
However, its necessarily to emphasize three points.
First, our proposed method only applies in cases where ambiguities exist in the definition. 
If the conditions of certain AI activities are explicitly defined in the license, then we should directly follow that.
Second, due to the various definitions adopted in different licenses, the final mappings depend on each specific case and may differ from Figure~\ref{fig:tax}.
Lastly, one AI activity may trigger multiple license conditions. For example, a fine-tuned model can be seen as a combination with weak separation of the original model, while it can also be viewed as a derivative from fine-tuning data.
Therefore, we should design a mechanism to trace these multi-source dependency structures in ML projects, which we will detail in the next section.
 
\begin{figure}[t]
    \centering
    \includegraphics[width=\linewidth]{fig/taxonomy.pdf}
    \caption{Our proposed taxonomy bridging AI activities and license language keywords based on their result forms.}
    \Description{}
    \label{fig:tax}
    %\vspace{-5mm}
\end{figure}

%translated, altered, arranged, transformed, or otherwise modified 
%Licensing Language Requires Standardization and  to ML and AI
%the notion of derivative work is ill defined
%conceptual ambiguities in existing licensing language
%There is no consensus on whether the use

\subsection{Structure of ML Projects}
ML projects have unique dependency relationships compared to OSS projects, like the dependencies between generated content and generation model, as well as between training data and trained model.
We can summarize these dependencies in ML projects into three categories:
\begin{itemize}%[leftmargin=*]
    \item \textbf{Mix-works} be embeded in the new work, either verbatim or in part, in a tangible form.
    They usually result from direct copying of original components or reusing them through AI activities like combination and amalgamation. These components are embedded into ML projects and must be released with the new work. 
    For example, if we release a new work utilizing Mixture of Experts (MoE), it is equivalent to releasing all weak learners.

    \item \textbf{Sub-works} are similar to mix-works, but the difference is that they are not embedded in the new work. For instance, if we manage to release MoE model along with the data used for training the gating network, then this data will be regarded as the sub-works of MoE model.
    
    \item \textbf{Aux-works} are components used to build the new work and are either included in it or released with it. For example, the original model used for knowledge distillation.
\end{itemize}

Figure~\ref{fig:stru} illustrates the structure of a work constructed by reusing multiple components in ML projects.
The final ML project may be constructed through iterative reuse of other works, resulting in a ternary dependencies tree for this project.
The reason we need this specially-designed tree structure is that works with different dependency types have different license conditions proliferation rules, as illustrated by dotted boxes in Figure~\ref{fig:stru}, which need to be handled separately during subsequent license analysis. 


\begin{figure}[t]
    \centering
    \includegraphics[width=\linewidth]{fig/structure.pdf}
    \caption{The proposed structure for capturing work dependencies in ML projects with multiple reused components.}
    \Description{}
    \label{fig:stru}
    %\vspace{-5mm}
\end{figure}

\subsection{License Analysis Steps for ML Projects}

The detailed implementation of the preparation steps and analysis step in ModelGo are as follows.

\textbf{Preparation Step 1}: Following our proposed taxonomy, we have manually transcribed the terms in the license text to a standard machine-readable file in YAML format\footnote{We attempted to use chatGPT to generate this content, but it often behaved unreliably in understanding our taxonomy and produced some stochastic answers~\cite{bender2021dangers}.}.
This file contain following informations for each license:
\begin{itemize}%[leftmargin=*]
    \item Basic license descriptions, including its full name, SPDX short ID, license version, license types (e.g., public domain, permissive, copyleft, proprietary), preferred work types (e.g., software, data, model), and supporting labels such as \textit{disclose code required} and \textit{auto-relicensing applied}.
    
    \item Rights granting information, including granted rights and reserved rights as defined by the license text, along with the permitted reusing methods and permitted result forms for redistribution.
    The prefix of such granting also be noted for cases where the granted rights can be revoked.

    \item Applicable terms for each AI activity, which contain result forms and relicensability of the activity, corresponding restrictions, and obligations. 
    This item will be marked as \textit{No Defined} if both the activity and the result forms of this activity are not explicitly covered in the license text.
\end{itemize}

\textbf{Preparation Step 2}:
To capture the dependency structure of works as shown in Figure~\ref{fig:stru}, we encode the rules of dependencies construction for each AI activity.
For example, if we generate embeddings of a corpus using an NN model, then the corpus is considered the sub-work of the generated embeddings, with the activity labeled as \textit{embed}, and the NN model is categorized as the aux-work with the activity labeled as \textit{use}.
Furthermore, if the corpus is a collection of smaller corpora, then these smaller corpuses are categorized as the mix-works of the integrated corpus, with the activity labeled as \textit{combine}.
By recursively traversing this dependencies tree, we can gather all the dependent works and the activities used to build this ML project.

It is important to emphasize a concept in our license analysis approach called \textit{activity proliferation}, which means that the activity performed by a work will recursively proliferate to all its mix-works.
In the example of the corpus collection mentioned above, the \textit{embed} calculation performed on the collection will be applied to all the smaller corpuses, triggering their license conditions related to \textit{embed} as well.
Similarly, as shown in Figure~\ref{fig:stru}, all rights granting validation and license conflicts analysis of a work should be proliferated to all its mix-works.
On the other hand, aux-works are not released with the project, so they are out of the scope of license conflict analysis and rights granting validation for release.
In summary, mix-works, sub-works, and aux-works have different scopes in ML license analysis, which is why we need to distinguish between them.

\textbf{Analysis Step}:
Given the license information and dependencies tree of ML projects, we are ready to analyze the license conflicts within it. 
ModelGo's license analysis consists of three phases:

\textit{Initial phase}, where we register each component with exact license name, version, type, and format (e.g., raw, binary, SaaS), and then construct their workflows using our predefined reusing functions to capture the dependencies.
The release policy should be preset here, and we support personal use, sharing, and selling. 
Normally, few conditions apply when you only use the work personally, and most license terms limit behaviors like redistribution, sublicensing, and commercial use.

\textit{License determination phase}, where we iteratively derive the eligible new licenses for intermediate reused results.
Copyleft proliferation occurs when there is a triggered copyleft license in the relied components.
An error will raise if there are other copyleft licenses or if there are components that cannot be relicensed.
To condense our analysis results, we prioritize using \textit{Unlicense} for intermediate results once they are relicenseable.
After this phase, all components and their derivatives should have a well-determined license name.

\textit{License validation phase}, where we validate the required rights for construct and release this project whether can be granted.
The validation also includes compliance with disclosure requirements, such as when a components is in binary format but subject to conditions that require source code disclosure.
The releaseability of the final result will be validated upon its mix-works and sub-works, and then an assessment report will be generated.


Table~\ref{tab:analysis} presents the warnings, errors, restrictions, obligations, and notices that can be detected using ModelGo.
Table~\ref{tab:list} lists the licenses supported by ModelGo, which collectively cover over 96\% of licensed models and datasets on Huggingface\footnote{No major changes between different version CCs, so they are all considered as supported. Licenses without clear names and versions are excluded from the calculation. Worth mentioning, our coverage represents only 24.8\% and 6.0\% of the models and datasets on the entire repository due to the significant number of works without license information.}.
In the next section, we will present five case studies based on real ML components.


\begin{table}[t]
    \caption{License warnings, errors, restrictions/obligations, and notices assessed by ModelGo in \textcolor{Permissive}{initial phase}, \textcolor{Copyleft}{license determination phase} and license validation phase.}
    %\vspace{-3mm}
    \scriptsize
    \label{tab:analysis}
    \begin{tabular}{|p{3.3cm}|p{4.3cm}|}

    \hline
    \rowcolor[gray]{.8}
    \textbf{Warning, Error, Restriction, Notice} & \textbf{Description} \\ \hline
    
    \textcolor{Permissive}{Copyleft / Revocable / No Public Notice} & This license or its granted rights are \textbf{copyleft / revocable / no public}. \\ \hline \hline
    
    \textcolor{Permissive}{License Type Mismatch Warning} & License preferred work type is \textbf{not compatible} with this work type. \\ \hline
    License Disclose Self Warning & License requires this work (in binary or SaaS format) to remain \textbf{open source} or provide a \textbf{readable copy} of the source code. \\ \hline
    Rights Not Granted Warning & License of this work does \textbf{not explicitly grant} you the right to do (...) \\ \hline \hline
    
    
    Rights Not Granted Error & License of this work \textbf{cannot grant} you the right to do (...) \\ \hline
    %\textcolor{Copyleft}{Multiple Copyleft Licenses Error} & Work has a license conflict as it involves \textbf{multiple copyleft} licenses. \\ \hline
    \textcolor{Copyleft}{License Incompatibility Error} & Work has a license conflict as it involves \textbf{multiple incompatible} licenses. \\ \hline
    \textcolor{Copyleft}{Cannot Relicense Error} & Work has a license conflict as it required \textbf{relicense} rights not be granted. \\ \hline
    Cannot Share Error & License \textbf{prohibits sharing} of this work. \\ \hline \hline

    State Changes Restriction & This work must \textbf{state changes} according to related license(s). \\ \hline
    Include License Restriction & This work must retain the \textbf{original license file} according to the related license(s). \\ \hline
    Include Notice Restriction & This work must retain all \textbf{notice files} (may contain copyright, patent, trademark and attribution) according to the related license(s). \\ \hline
    Use Behavioral Restriction & This work must comply with the \textbf{use restriction} terms according to related license(s). \\ \hline
    Runtime Restriction & This work must comply with the \textbf{rumtime restriction} terms according to related license(s). \\ \hline

    \end{tabular}
    %\vspace{-2mm}
\end{table}


\begin{comment}

Apache-2.0, Unlicense, MIT, AFL-3.0, GPL-3.0, AGPL-3.0, LGPL-3.0, LGPL-2.1, BSD-3-Clause, BSD-3-Clause-Clear, BSD-2-Clause, WTFPL-2.0, OSL-3.0, ECL-2.0, CC0-1.0, CC-BY-4.0, CC-BY-SA-4.0, CC-BY-NC-4.0, CC-BY-ND-4.0, CC-BY-NC-ND-4.0, CC-BY-NC-SA-4.0, PDDL, C-UDA, LGPL-LR, GFDL, OpenRAIL++, CreativeML-OpenRAIL-M, BigScience-BLOOM-RAIL-1.0, OPT-175B, SEER, Llama2

If CC SA-licensed content is included in a database, does the entire database have to be licensed under an SA license?

If CC SA-licensed content is included in a database, does the entire database have to be licensed under an SA license?
CC licenses never require a reuser of a CC-licensed work to make the original work or resulting works (collections, derivatives, etc.) publicly available. There are lots of private reuses of works that are permitted by CC’s licenses that do not require compliance with their terms. Regarding ShareAlike, the condition only applies if a work is modified and if the work is shared publicly. In the situation where a reuser created a dataset of photos and made it publicly available, and assuming copyright permission is required, then what is released is likely a collection or compilation of pre-existing works. CC licenses do not require the collection or the compilation itself to be made available under an SA license, even though each individual work is still licensed individually under an SA license and if they were modified by the distributor the modified photo would need to be licensed under the same terms. For example, were Creative Commons to compile photographs from a photo sharing website under a BY-SA 2.0 license and create a database that it then publicly distributed, CC could license the collection as a whole under a BY license, but the photographs would continue to be licensed under BY-SA 2.0.
\end{comment}