\section{Preliminary License Analysis Results}
\label{sec:case}
This section seeks to answer a key question: \emph{Should I continue using traditional OSS and free-content licenses to publish my model, and what are the associated risks?}
To explore this question, we evaluate commonly adopted licenses in the context of model publishing by MG Analyzer to assess whether they are still effective as intended.
The example workflow involves combining two models and publishing them as a service, as shown in Figure~\ref{fig:case1}.
Here, we consider two scenarios: 1) publishing the combined model as a service; 2) publishing the data generated from the combined model.
Models A and B are non-binding works that correspond to the settings in Table~\ref{tab:result}, with their respective analysis results also presented in the table.
Model C is an intermediate work created by combining A and B, and Data D is the generated output from C. 
Work E involves republishing C as a service with the intent to sell, while Work F involves republishing D as a literary form, also with the intent to sell.
To be more convincing, we use real-world models and their respective licenses for demonstration.

\begin{figure}[t]
    \centering
    \includegraphics[]{fig/case1.pdf}
    \caption{Example Workflow: Combine Models, Then Publish.}
    \Description{}
    \label{fig:case1}
    \vspace{-3mm}
\end{figure}

\begin{table}[]
    \caption{Settings and Corresponding Analysis Results from MG Analyzer with Fuzz Form Matching Enabled.}
    \footnotesize
    \vspace{-2mm}
    \begin{tabular}{|ll|}
    \hline \rowcolor[gray]{.8}
    \multicolumn{1}{|l|}{\textbf{Work: License}} & \textbf{Work: Report code} \\ \hline
    \multicolumn{1}{|p{3.2cm}|}{(i) C: AGPL-3.0. D: Unlicense. \newline (ii) D: Unlicense } & \multicolumn{1}{p{4.5cm}|}{ (i) E: N1N2N3$\times$2; \textbf{W5}$\times$2; W1$\times$3. F: W1$\times$3. \newline (ii) E: W1$\times$2. F: W1.} \\ \hline
    \multicolumn{2}{|p{8cm}|}{\textbf{OSS License Setting:} \newline
    (i) A$\leftarrow$PhoBERT~\cite{nguyen2020phobert}: AGPL-3.0. B$\leftarrow$CKIP-Transformers~\cite{lin2022hantrans}: GPL-3.0. \newline
    (ii) A$\leftarrow$PhoBERT~\cite{nguyen2020phobert}: AGPL-3.0. B$\leftarrow$\emph{None}.
    }\\ \hline \hline

    \multicolumn{1}{|p{3.2cm}|}{C: Unlicense. D: Unlicense.} & E: W1$\times$2; \textbf{E2}$\times$2; \textbf{E5}$\times$2. F: W1$\times$2 . \\ \hline
    \multicolumn{2}{|p{8cm}|}{\textbf{Free-content License Setting:}\newline
        A$\leftarrow$ MPT-Chat~\cite{MosaicML2023Introducing}: CC-BY-NC-SA-4.0.  B$\leftarrow$Command R+~\cite{cohereforai2024c4ai}: CC-BY-NC-4.0.
    } \\ \hline \hline

    \multicolumn{1}{|p{3.2cm}|}{(iii) D: Unlicense. \newline (iv) D: Unlicense.} & \multicolumn{1}{p{4.5cm}|}{ (iii) E: N1; N2; \textbf{W5} . F: \emph{None}. \newline (iv) E: N1; N2; W2$\times$2; \textbf{E2}; \textbf{E5}. F: W2, \textbf{E5}.} \\ \hline
    \multicolumn{2}{|p{8cm}|}{\textbf{Model License Setting:}\newline
        (iii) A$\leftarrow$ MG-BY-OS. B$\leftarrow$\emph{None}. (iv) A$\leftarrow$ MG-BY-NC. B$\leftarrow$\emph{None}.
    } \\ \hline
    \end{tabular}
    \label{tab:result}
    \vspace{-3mm}
\end{table}

First, we evaluate two OSS licenses: AGPL-3.0 and GPL-3.0, which are considered enforceable open-source licenses with copyleft clauses.
In setting (i), Work E triggers two \emph{Disclose Source Code} warnings (code W5, refer to Table~\ref{tab:report} for code definitions), and AGPL-3.0 successfully proliferates\footnote{While two copyleft conditions are simultaneously triggered here, they can be resolved because GPL-3.0 is compatible with AGPL-3.0. Appendix~\ref{apdx:grapher} visualizes this workflow.} to Work C.
However, with a small adjustment, we can circumvent these clauses by republishing the generated content rather than providing the model as a service. 
As demonstrated by Work D, there is no W5 warning, and the content is licensed under the Unlicense.
Furthermore, the condition in AGPL-3.0 that triggers the disclose code clauses related to remote access is \emph{you modify the Program}, which means you can directly republish copies as a service to circumvent this clause.
As reflected in setting (ii), there are no more W5 warnings, only two warnings related to the non-standard licensing remain.

Second, we evaluate two free-content licenses: CC-BY-NC-SA-4.0 and CC-BY-NC-4.0, both of which prohibit the commercial use of the governed work. 
As shown in the results, the republication of Work E successfully triggers E2 errors because the rights to commercial use are reserved. 
However, we can still circumvent these clauses by generating and then sharing the output, as these licenses lack rules regarding the generated work.

Third, we evaluate MG Licenses: MG-BY-OS and MG-BY-NC, which contain open sourcing and non-commercial use clauses, respectively.
In setting (iii), our MG-BY-OS license successfully triggers the W5 warning, indicating that Work E must disclose its source code.
In setting (iv), the non-commercial use error E5 is reported by the generated Work F.
As a model license, we do not enforce licensing on generated content, allowing Work D to be licensed under the Unlicense, albeit with certain restrictions.
A summary of the rights granted and restrictions imposed by these licenses can be found in their \emph{Model Sheet} provided in Appendix~\ref{apdx:model_sheet}.

It is worth mentioning that all results were obtained with \emph{fuzzy form matching} enabled, maximizing the detection of potential risks.
If these \emph{fuzz rules} were disabled, fewer issues would be reported. 
Furthermore, our MG Analyzer is designed to help developers be aware of potential compliance issues in ML projects. 
Its results should not be considered legal advice or a defense in dispute resolution. Please refer to our disclaimers in Appendix~\ref{apdx:disclamers}.
The most promising application lies in providing license analysis within complex ML workflows. An in-the-wild case can be found in Appendix~\ref{apdx:case2}.

\begin{tcolorbox}[boxrule=0.2mm, boxsep=1mm, left=0mm, right=0mm, top=0.1mm, bottom=0.1mm]
    \textbf{Summary}: GPL, AGPL, and CC licenses can be easily circumvented, leading to unintended misuse of the ML models they govern. In contrast, MG Licenses offer greater clarity and flexibility tailored to various publishing scenarios, promoting a more standardized and transparent approach to model licensing.
\end{tcolorbox}